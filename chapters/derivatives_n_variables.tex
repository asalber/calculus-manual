% Author: Alfredo Sánchez Alberca (asalber@ceu.es)

\section{Several Variables Differentiable Calculus}

\mode<presentation>{
%---------------------------------------------------------------------slide----
\begin{frame}
\frametitle{Several variables differentiable calculus}
\tableofcontents[sectionstyle=show/hide,hideothersubsections]
\end{frame}
}


\subsection{Vector functions of a single real variable}

% ---------------------------------------------------------------------slide----
\begin{frame}
\frametitle{Vector function of a single real variable}

\begin{definition}[Vector function of a single real variable]
A \emph{vector function of a single real variable} or \emph{vector field of a scalar variable} is a function that maps every scalar value $t\in D\subseteq \mathbb{R}$ into a vector $(x_1(t),\ldots,x_n(t))$ in $\mathbb{R}^n$:
\[
\begin{array}{rccl}
f: & \mathbb{R} & \longrightarrow & \mathbb{R}^n\\
& t & \longrightarrow & (x_1(t),\ldots, x_n(t))
\end{array}
\]
where $x_i(t)$, $i=1,\ldots,n$, are real function of a single real variable known as \emph{coordinate functions}.
\end{definition}

The most common vector field of scalar variable are in the the real plane $\mathbb{R}^2$, where usually they are represented as
\[
f(t)=x(t)\mathbf{i}+y(t)\mathbf{j},
\]
and in the real space $\mathbb{R}^3$, where usually they are represented as
\[
f(t)=x(t)\mathbf{i}+y(t)\mathbf{j}+z(t)\mathbf{k},
\]
\end{frame}


% ---------------------------------------------------------------------slide----
\begin{frame}
\frametitle{Graphic representation of vector fields}

\begin{columns}
\begin{column}{0.5\textwidth}
The graphic representation of a vector field in $\mathbb{R}^2$ is a trajectory in the real plane.
\begin{center}
\tikzsetnextfilename{derivatives_n_variables/trajectory_plane}
\input{img/derivatives_n_variables/trajectory_plane}
\end{center}
\end{column}
\begin{column}{0.5\textwidth}
The graphic representation of a vector field in $\mathbb{R}^3$ is a trajectory in the real space.
\begin{center}
\tikzsetnextfilename{derivatives_n_variables/trajectory_space}
\input{img/derivatives_n_variables/trajectory_space}
\end{center}
\end{column}
\end{columns}
\end{frame}



% ---------------------------------------------------------------------slide----
\begin{frame}
\frametitle{Derivative of a vector field}
The concept of derivative as the limit of the average rate of change of a function can be extended easily to vector fields.

\begin{definition}[Derivative of a vectorial field]
A vectorial field $f(t)=(x_1(t),\ldots,x_n(t))$ is \emph{differentiable} at a point $t=a$ if the limit
\[
\lim_{\Delta t\rightarrow 0} \frac{f(a+\Delta t)-f(a)}{\Delta t}.
\]
exists.
In such a case, the value of the limit is known as the \emph{derivative} of the vector field at $a$, and it is written $f'(a)$.
\end{definition}
\end{frame}



% ---------------------------------------------------------------------slide----
\begin{frame}
\frametitle{Derivative of a vector field}
Many properties of real functions of a single real variable can be extended to vector fields through its component functions.
Thus, for instance, the derivative of a vector field can be computed from the derivatives of its component functions.

\begin{theorem}
Given a vector field $f(t)=(x_1(t),\ldots,x_n(t))$, if $x_i(t)$ is differentiable at $t=a$ for all $i=1,\ldots,n$, then $f$ is differentiable at $a$ and its derivative is
\[
f'(a)=(x_1'(a),\ldots,x_n'(a))
\]
\end{theorem}

The proof for a vectorial field in $\mathbb{R}^2$ is easy.
\begin{align*}
f'(a)&=\lim_{\Delta t\rightarrow 0} \frac{f(a+\Delta t)-f(a)}{\Delta t} = \lim_{\Delta t\rightarrow 0} \frac{(x(a+\Delta t),y(a+\Delta t))-(x(a),y(a))}{\Delta t} =\\
&=  \lim_{\Delta t\rightarrow 0} \left(\frac{x(a+\Delta t)-x(a)}{\Delta t},\frac{y(a+\Delta t)-y(a)}{\Delta t}\right) =\\
&= \left(\lim_{\Delta t\rightarrow 0}\frac{x(a+\Delta t)-x(a)}{\Delta t},\lim_{\Delta t\rightarrow 0}\frac{y(a+\Delta t)-y(a)}{\Delta t}\right) =
(x'(a),y'(a)).
\end{align*}
\end{frame}


% ---------------------------------------------------------------------slide----
\begin{frame}
\frametitle{Kinematics: Curvilinear motion}
The notion of derivative as a velocity along a trajectory in the real line can be generalized to a trajectory in any euclidean space $\mathbb{R}^n$.

In case of a two dimensional space $\mathbb{R}^2$, if $f(t)$ describes the position of a moving object in the real plane at any time $t$, taking as reference the coordinates origin $O$ and the unitary vectors $\{\mathbf{i}=(1,0),\mathbf{j}=(0,1)\}$, we can represent the position of the moving object $P$ at every moment $t$ with a vector $\vec{OP}=x(t)\mathbf{i}+y(t)\mathbf{j}$, where the coordinates
\[
\begin{cases}
x=x(t)\\
y=y(t)
\end{cases}
\quad
t\in \mbox{Dom}(f)
\]
are the \emph{coordinate functions} of $f$.

\begin{center}
\tikzsetnextfilename{derivatives_1_variable/curvilinear_motion}
\mode<article>{\input{img/derivatives_n_variables/curvilinear_motion}}
\mode<presentation>{\scalebox{0.8}{\input{img/derivatives_n_variables/curvilinear_motion}}}
\end{center}
\end{frame}


% ---------------------------------------------------------------------slide----
\begin{frame}
\frametitle{Velocity of a curvilinear motion in the plane}
In this context the derivative of a trajectory $f'(a)=(x_1'(a),\ldots,x_n'(a))$ is the \emph{velocity} vector of the trajectory $f$ at moment $t=a$.

\structure{\bfseries Example}
Given the trajectory $f(t) = (\cos t,\sin t)$, $t\in \mathbb{R}$, whose image is the unit circumference centered in the coordinate origin, its coordinate functions are $x(t) = \cos t$, $y(t) = \sin t$, $t\in \mathbb{R}$, and its velocity is
\[
\mathbf{v}=f'(t)=(x'(t),y'(t))=(-\sin t, \cos t).
\]
In the moment $t=\pi/4$, the object is in position $f(\pi/4) = (\cos(\pi/4),\sin(\pi/4)) =(\sqrt{2}/2,\sqrt{2}/2)$
and it is moving with a velocity $\mathbf{v}=f'(\pi/4)=(-\sin(\pi/4),\cos(\pi/4))=(-\sqrt{2}/2,\sqrt{2}/2)$.
\begin{center}
\tikzsetnextfilename{derivatives_1_variable/circumference_trajectory}
\mode<article>{% Author: Alfredo Sánchez Alberca (asalber@ceu.es)
\begin{tikzpicture}
\begin{axis}[
    2dfun, 
    xmin=-1.5, xmax=1.5,
    ymin=-1.5, ymax=1.5,
    axis equal=true,  
    height=3cm,
    clip=false,
    ]
	\addplot+[domain=0:2*pi, samples=200, smooth]({cos(deg(x))},{sin(deg(x))});
	\coordinate (P) at (0.7071068,0.7071068); 
	\fill (P) circle (1.2pt) node [anchor=south west] {$P=f(\pi/4)$};
	\draw [->, color2] (P) -- (0,1.4142);
\end{axis};
\end{tikzpicture}
}
\mode<presentation>{\scalebox{0.9}{% Author: Alfredo Sánchez Alberca (asalber@ceu.es)
\begin{tikzpicture}
\begin{axis}[
    2dfun, 
    xmin=-1.5, xmax=1.5,
    ymin=-1.5, ymax=1.5,
    axis equal=true,  
    height=3cm,
    clip=false,
    ]
	\addplot+[domain=0:2*pi, samples=200, smooth]({cos(deg(x))},{sin(deg(x))});
	\coordinate (P) at (0.7071068,0.7071068); 
	\fill (P) circle (1.2pt) node [anchor=south west] {$P=f(\pi/4)$};
	\draw [->, color2] (P) -- (0,1.4142);
\end{axis};
\end{tikzpicture}
}}
\end{center}
Observe that the module of the velocity vector is always 1 as
$|\mathbf{v}|=\sqrt{(-\sin t)^2+(\cos t)^2}=1$.
\end{frame}



\subsection{Tangent line to a trajectory}
% ---------------------------------------------------------------------slide----
\begin{frame}
\frametitle{Tangent line to a trajectory in the plane}
\framesubtitle{Vectorial equation}
Given a trajectory $f(t)$ in the real plane, the vectors that are parallel to the velocity $\mathbf{v}$ at a moment $a$ are called \emph{tangent vectors} to the trajectory $f$ at the moment $a$, and the line passing through $P=f(a)$ directed by $\mathbf{v}$ is the tangent line to the graph of $f$ at the moment $a$.

\begin{definition}[Tangent line to a trajectory]
Given a trajectory $f(t)$ in the real plane $\mathbb{R}^2$, the \emph{tangent line} to to the graph of $f$ at $a$ is the line with equation
\begin{align*}
l:(x,y) &= f(a)+tf'(a) = (x(a),y(a))+t(x'(a),y'(a))\\
& = (x(a)+tx'(a),y(a)+ty'(a)).
\end{align*}
\end{definition}
\end{frame}


% ---------------------------------------------------------------------slide----
\begin{frame}
\frametitle{Tangent line to a trajectory in the plane}
\framesubtitle{Example}
We have seen that for the trajectory $f(t) = (\cos t,\sin t)$, $t\in \mathbb{R}$, whose image is the unit circumference centered at the coordinate origin, the object position at the moment $t=\pi/4$ is $f(\pi/4)=(\sqrt{2}/2,\sqrt{2}/2)$ and its velocity $\mathbf{v}=(-\sqrt{2}/2,\sqrt{2}/2)$.
Thus the equation of the tangent line to $f$ at that moment is
\begin{align*}
l: (x,y) & = f(\pi/4)+t\mathbf{v} =
\left(\frac{\sqrt{2}}{2},\frac{\sqrt{2}}{2}\right)+t\left(\frac{-\sqrt{2}}{2},\frac{\sqrt{2}}{2}\right) =\\
& =\left(\frac{\sqrt{2}}{2}-t\frac{\sqrt{2}}{2},\frac{\sqrt{2}}{2}+t\frac{\sqrt{2}}{2}\right).
\end{align*}

\end{frame}


% ---------------------------------------------------------------------slide----
\begin{frame}
\frametitle{Tangent line to a trajectory in the plane}
\framesubtitle{Cartesian and point-slope equations}
From the vectorial equation of the tangent to a trajectory $f(t)$ at the moment $t=a$ we can get the coordinate functions
\[
\begin{cases}
x=x(a)+tx'(a)\\
y=y(a)+ty'(a)
\end{cases}
\quad t\in \mathbb{R},
\]
and solving for $t$ and equalling both equations we get the \emph{Cartesian equation} of the tangent
\[
\frac{x-x(a)}{x'(a)}=\frac{y-y(a)}{y'(a)},
\]
if $x'(a)\neq 0$ and $y'(a)\neq 0$.

From this equation it is easy to get the \emph{point-slope equation} of the tangent
\[
y-y(a)=\frac{y'(a)}{x'(a)}(x-x(a)).
\]
\end{frame}


% ---------------------------------------------------------------------slide----
\begin{frame}
\frametitle{Tangent line to a trajectory in the plane}
\framesubtitle{Example of Cartesian and point-slope equations}
Using the vectorial equation of the tangent of the previous example
\[
l: (x,y)=\left(\frac{\sqrt{2}}{2}-t\frac{\sqrt{2}}{2},\frac{\sqrt{2}}{2}+t\frac{\sqrt{2}}{2}\right),
\]
its Cartesian equation is
\[
\frac{x-\sqrt{2}/2}{-\sqrt{2}/2} = \frac{y-\sqrt{2}/2}{\sqrt{2}/2}
\]
and the point-slope equation is
\[
y-\sqrt{2}/2 = \frac{-\sqrt{2}/2}{\sqrt{2}/2}(x-\sqrt{2}/2) \Rightarrow y=-x+\sqrt{2}.
\]
\end{frame}


% ---------------------------------------------------------------------slide----
\begin{frame}
\frametitle{Normal line to a trajectory in the plane}
We have seen that the tangent line to a trajectory $f(t)$ at $a$ is the line passing through the point
$P=f(a)$ directed by the velocity vector $\mathbf{v}=f'(a)=(x'(a),y'(a))$.
If we take as direction vector a vector orthogonal to $\mathbf{v}$, we get another line that is known as \emph{normal line} to the trajectory.
\begin{definition}[Normal line to a trajectory]
Given a trajectory $f(t)$ in the real plane $\mathbb{R}^2$, the \emph{normal line} to the graph of $f$ at moment $t=a$ is the line with equation
\[
l: (x,y)=(x(a),y(a))+t(y'(a),-x'(a)) = (x(a)+ty'(a),y(a)-tx'(a)).
\]
\end{definition}
The Cartesian equation is
\[
\frac{x-x(a)}{y'(a)} = \frac{y-y(a)}{-x'(a)},
\]
and the point-slope equation is
\[
y-y(a) = \frac{-x'(a)}{y'(a)}(x-x(a)).
\]
The normal line is always perpendicular to the tangent line as their direction vectors are orthogonal.
\end{frame}


% ---------------------------------------------------------------------slide----
\begin{frame}
\frametitle{Normal line to a trajectory in the plane}
\framesubtitle{Example}
Considering again the trajectory of the unit circumference $f(t) = (\cos t,\sin t)$, $t\in \mathbb{R}$, the normal line to the graph of $f$ at moment $t=\pi/4$ is
\begin{align*}
l: (x,y)&=(\cos(\pi/2),\sin(\pi/2))+t(\cos(\pi/2),\sin(\pi/2)) =\\
&= \left(\frac{\sqrt{2}}{2},\frac{\sqrt{2}}{2}\right)+t\left(\frac{\sqrt{2}}{2},\frac{\sqrt{2}}{2}\right)
=\left(\frac{\sqrt{2}}{2}+t\frac{\sqrt{2}}{2},\frac{\sqrt{2}}{2}+t\frac{\sqrt{2}}{2}\right),
\end{align*}
the Cartesian equation is
\[
\frac{x-\sqrt{2}/2}{\sqrt{2}/2} = \frac{y-\sqrt{2}/2}{\sqrt{2}/2},
\]
and the point-slope equation is
\[
y-\sqrt{2}/2 = \frac{\sqrt{2}/2}{\sqrt{2}/2}(x-\sqrt{2}/2) \Rightarrow y=x.
\]
% \begin{center}
% \scalebox{0.8}{\input{img/calculo_diferencial_1_variable/circunferencia_tangente_normal}}
% \end{center}
\end{frame}


% ---------------------------------------------------------------------slide----
\begin{frame}
\frametitle{Tangent and normal lines to a function}
A particular case of tangent and normal lines to a trajectory are the tangent and normal lines to a function of one real variable.
For every function $y=f(x)$, the trajectory that trace its graph is
\[
g(x) = (x,f(x))  \quad x\in \mathbb{R},
\]
and its velocity is
\[
g'(x) = (1,f'(x)),
\]
so that the tangent line to $g$ at the moment $a$ is
\[
\frac{x-a}{1} = \frac{y-f(a)}{f'(a)} \Rightarrow y-f(a) = f'(a)(x-a),
\]
and the normal line is
\[
\frac{x-a}{f'(a)} = \frac{y-f(a)}{-1} \Rightarrow y-f(a) = \frac{-1}{f'(a)}(x-a),
\]
\end{frame}


% ---------------------------------------------------------------------slide----
\begin{frame}
\frametitle{Tangent and normal lines to a function}
\framesubtitle{Example}
Given the function $y=x^2$, the trajectory that traces its graph is $g(x)=(x,x^2)$ and its velocity is
$g'(x)=(1,2x)$. At the moment $x=1$ the trajectory passes through the point $(1,1)$ with a velocity $(1,2)$.
Thus, the tangent line at that moment is
\[
\frac{x-1}{1} = \frac{y-1}{2} \Rightarrow y-1 = 2(x-1) \Rightarrow y = 2x-1,
\]
and the normal line is
\[
\frac{x-1}{2} = \frac{y-1}{-1} \Rightarrow y-1 = \frac{-1}{2}(x-1) \Rightarrow y = \frac{-x}{2}+\frac{3}{2}.
\]
\begin{center}
\tikzsetnextfilename{derivatives_n_variables/parabola_tangent_normal}
\input{img/derivatives_n_variables/parabola_tangent_normal}
\end{center}
\end{frame}


% ---------------------------------------------------------------------slide----
\begin{frame}
\frametitle{Tangent line to a trajectory in the space}
The concept of tangent line to a trajectory can be easily extended from the real plane to the three-dimensional space $\mathbb{R}^3$.

If $f(t)=(x(t),y(t),z(t))$, $t\in \mathbb{R}$, is a trajectory in the real space $\mathbb{R}^3$, then at the moment $a$, the moving object that follows this trajectory will be at the position $P=(x(a),y(a),z(a))$ with a velocity $\mathbf{v}=f'(t)=(x'(t),y'(t),z'(t))$.
Thus, the tangent line to $f$ at this moment have the following vectorial equation
\begin{align*}
l&: (x,y,z)=(x(a),y(a),z(a))+t(x'(a),y'(a),z'(a)) =\\
&= (x(a)+tx'(a),y(a)+ty'(a),z(a)+tz'(a)),
\end{align*}
and the Cartesian equations are
\[
\frac{x-x(a)}{x'(a)}=\frac{y-y(a)}{y'(a)}=\frac{z-z(a)}{z'(a)},
\]
provided that $x'(a)\neq 0$, $y'(a)\neq 0$ y $z'(a)\neq 0$.
\end{frame}


% ---------------------------------------------------------------------slide----
\begin{frame}
\frametitle{Tangent line to a trajectory in the space}
\framesubtitle{Example}
Given the trajectory $f(t)=(\cos t, \sin t, t)$, $t\in \mathbb{R}$ in the real space, at the moment $t=\pi/2$ the trajectory passes through the point
\[
f(\pi/2)=(\cos(\pi/2),\sin(\pi/2),\pi/2)=(0,1,\pi/2),
\]
with velocity
\[
\mathbf{v}=f'(\pi/2)=(-\sin(\pi/2),\cos(\pi/2), 1)=(-1,0,1),
\]
and the tangent line to the graph of $f$ at that moment is
\[
l:(x,y,z)=(0,1,\pi/2)+t(-1,0,1) = (-t,1,t+\pi/2).
\]
\begin{center}
\tikzsetnextfilename{derivatives_n_variables/tangent_trajectory_space}
\input{img/derivatives_n_variables/tangent_trajectory_space}
\end{center}
\end{frame}


% ---------------------------------------------------------------------slide----
\begin{frame}
\frametitle{Normal plane to a trajectory in the space}
In the three-dimensional space $\mathbb{R}^3$, the normal line to a trajectory is not unique.
There are an infinite number of normal lines and all of them are in the normal plane.

If $f(t)=(x(t),y(t),z(t))$, $t\in \mathbb{R}$, is a trajectory in the real space $\mathbb{R}^3$, then at the moment $a$, the moving object that follows this trajectory will be at the position $P=(x(a),y(a),z(a))$ with a velocity $\mathbf{v}=f'(t)=(x'(t),y'(t),z'(t))$.
Thus, using the velocity vector as normal vector the normal plane to $f$ at this moment have the following vectorial equation
\begin{align*}
\Pi &: (x-x(a),y-y(a),z-z(a))(x'(a),y'(a),z'(a)) = 0\\
&= x'(a)(x-x(a))+y'(a)(y-y(a))+z'(a)(z-z(a))=0.
\end{align*}
\end{frame}


% ---------------------------------------------------------------------slide----
\begin{frame}
\frametitle{Tangent line to a trajectory in the space}
\framesubtitle{Example}
For the trajectory of the previous example $f(t)=(\cos t, \sin t, t)$, $t\in \mathbb{R}$, at the moment $t=\pi/2$ the trajectory passes through the point
\[
f(\pi/2)=(\cos(\pi/2),\sin(\pi/2),\pi/2)=(0,1,\pi/2),
\]
with velocity
\[
\mathbf{v}=f'(\pi/2)=(-\sin(\pi/2),\cos(\pi/2), 1)=(-1,0,1),
\]
and normal plane to the graph of $f$ at that moment is
\[
\Pi:\left(x-0,y-1,z-\frac{\pi}{2}\right)(-1,0,1) =0 \Leftrightarrow -x+z-\frac{\pi}{2}=0.
\]
\begin{center}
\tikzsetnextfilename{derivatives_n_variables/normal_plane_trajectory_space}
\input{img/derivatives_n_variables/normal_plane_trajectory_space}
\end{center}
\end{frame}


\subsection{Functions of several variables}
%---------------------------------------------------------------------slide----
\begin{frame}
\frametitle{Need for functions of several variables}
A lot of problems in Geometry, Physics, Chemistry, Biology, etc. involve a variable that depend on two or more variables:
\begin{itemize}
\item The area of a triangle depends on two variables that are the base and height lengths.
\item The volume of a perfect gas depends on two variables that are the pressure and the temperature.
\item The way travelled by an object free falling depends on a lot of variables: the time, the area of the cross section of the object, the latitude and longitude of the object, the height above the sea level, the air pressure, the air temperature, the speed of wind, etc.
\end{itemize}
These dependencies are expressed with functions of several variables.
\end{frame}


%---------------------------------------------------------------------slide----
\begin{frame}
\frametitle{Functions of several real variables}
\begin{definition}[Functions of several real variables]
A \emph{function of $n$ real variables} or a \emph{scalar field} from a set $A_1\times \cdots \times A_n\subseteq \mathbb{R}^n$ in a set $B\subseteq \mathbb{R}$, is a relation that maps any tuple $(a_1,\ldots,a_n)\in A_1\times \cdots\times A_n$ into a unique element of $B$, denoted by $f(a_1,\ldots,a_n)$, that is knwon as the \emph{image} of $(a_1,\ldots,a_n)$ by $f$.
\[
\begin{array}{lccc}
f: & A_1\times\cdots\times A_n & \longrightarrow & B\\
   &(a_1,\ldots,a_n) & \longrightarrow & f(a_1,\ldots,a_n)
\end{array}
\]
\end{definition}

\structure{\textbf{Example}}
\begin{itemize}
\item The area of a triangle is a real function of two real variables
\[ f(x,y)=\frac{xy}{2}.\]
\item The volume of a perfect gas is a real function of two real variables
\[
v=f(t,p)=\frac{nRt}{p},\quad \mbox{with $n$ and $R$ constants.}
\]
\end{itemize}
\end{frame}


% ---------------------------------------------------------------------slide----
\begin{frame}
\frametitle{Graph of a function of two variables}
The graph of a function of two variables $f(x,y)$ is a surface in the real space $\mathbb{R}^3$ where every point of the surface has coordinates $(x,y,z)$, with $z=f(x,y)$.
\begin{center}
\tikzsetnextfilename{derivatives_n_variables/paraboloid}
%\href{https://www.google.es/search?q=x^2\%2By^2}{
\input{img/derivatives_n_variables/paraboloid}
%}
\end{center}
\end{frame}


%---------------------------------------------------------------------slide----
\begin{frame}
\frametitle{Graph of a function of two variables}
\framesubtitle{Area of a triangle}
The function $f(x,y)=\dfrac{xy}{2}$ that measures the area of a triangle of base $x$ and height $y$ has the graph below.
\begin{center}
\tikzsetnextfilename{derivatives_n_variables/area_triangle_graph}
%\href{https://www.google.es/search?q=x*y\%2F2}{
% Author: Alfredo Sánchez Alberca (asalber@ceu.es)
\begin{tikzpicture}
  \begin{axis}[view={120}{20},
  3dfun,
  xmin=-3, xmax=3,
  ymin=-3, ymax=3,
  zmin=-4, zmax=4,
  axis equal=true,
  mesh/interior colormap name=incolormap,
  colormap name=outcolormap,
%   every axis x label/.style={at={(xticklabel* cs:1.1)}},
%   every axis y label/.style={at={(yticklabel* cs:1.1)}},
%   every axis z label/.style={at={(zticklabel* cs:1.1)}},
%   x tick label style={anchor=east, inner sep=2pt}, 
%   y tick label style={anchor=north, inner sep=5pt},
%   z tick label style={anchor=east, inner sep=2pt},  
  clip=false,
  height=4cm,
  ]
  \addplot3[surf, domain=-3:3, y domain=-3:3, opacity=0.5, faceted color=color1] {x*y/2};
  \end{axis}
\end{tikzpicture}

%}
\end{center}
\end{frame}


%---------------------------------------------------------------------slide----
\begin{frame}
\frametitle{Graph of a function of two variables}
\framesubtitle{Wave of a water drop}
The function $\displaystyle f(x,y)=\frac{\sin(x^2+y^2)}{\sqrt{x^2+y^2}}$ has the peculiar graph below.
\begin{center}
\tikzsetnextfilename{derivatives_n_variables/water_drop_graph}
%\href{https://www.google.es/webhp?q=sin(sqrt(x^2\%2By^2))\%2Fsqrt(x^2\%2By^2)}{
% Author: Alfredo Sánchez Alberca (asalber@ceu.es)
\begin{tikzpicture}
  \begin{axis}[view={30}{20},
  3dfun,
  xmin=-13, xmax=13,
  ymin=-13, ymax=13,
  zmin=-2, zmax=10,
  axis equal=true,
  mesh/interior colormap name=incolormap,
  colormap name=outcolormap,
%   every axis x label/.style={at={(xticklabel* cs:1.1)}},
%   every axis y label/.style={at={(yticklabel* cs:1.1)}},
%   every axis z label/.style={at={(zticklabel* cs:1.1)}},
%   x tick label style={anchor=east, inner sep=2pt}, 
%   y tick label style={anchor=north, inner sep=5pt},
%   z tick label style={anchor=east, inner sep=2pt},  
  clip=false,
  height=4cm,
  ]
  \addplot3[surf, domain=-13:13, y domain=-13:13, opacity=0.5, faceted color=color1, samples=40] {10*sin(deg(sqrt(x^2+y^2)))/sqrt(x^2+y^2)};
  \end{axis}
\end{tikzpicture}

%}
\end{center}
\end{frame}


% ---------------------------------------------------------------------slide----
\begin{frame}
\frametitle{Level set of a scalar field}
\begin{definition}[Level set]
Given a scalar field $f:\mathbb{R}^n\rightarrow \mathbb{R}$, the \emph{level set} $c$ of $f$ is the set
\[
C_{f,c}=\{(x_1,\ldots,x_n): f(x_1,\ldots,x_n)=c\},
\]
that is, a set where the function takes on the constant value $c$.
\end{definition}
\end{frame}


% ---------------------------------------------------------------------slide----
\begin{frame}
\frametitle{Level set of a scalar field}
\framesubtitle{Example}
Given the scalar field $f(x,y)=x^2+y^2$ and the point $P=(1,1)$, the level set of $f$ that includes $P$ is
\[
C_{f,2} = \{(x,y): f(x,y)=f(1,1)=2\} = \{(x,y): x^2+y^2=2\},
\]
that is the circumference of radius $\sqrt{2}$ centered at the origin.
\begin{center}
\tikzsetnextfilename{derivatives_n_variables/paraboloid_level_set}
% Author: Alfredo Sánchez Alberca (asalber@ceu.es)
\begin{tikzpicture}
  \begin{axis}[view={120}{20},
  3dfun,
  xmin=-1.8, xmax=1.8,
  ymin=-1.8, ymax=1.8,
  zmin=-0.2, zmax=4,
%  axis equal=true,
  mesh/interior colormap name=incolormap,
  colormap name=outcolormap,
  clip=false,
  height=4cm,
  ]
  \addplot3[surf, domain=-1.5:1.5, y domain=-1.5:1.5, opacity=0.5, faceted color=color1] {x^2+y^2};
%  \addplot3[domain=-1.5:1.5, y domain=-1.5:1.5, contour gnuplot = {number=2, labels={false}, draw color = color2}, samples = 21,]{x^2+y^2};
  \coordinate (O) at (0,0,0);
  \coordinate (P) at (1,1,2);
  \fill (P) circle (1.2pt);
  \coordinate (Pxy) at (1,1,0);
  \fill (Pxy) circle (1.2pt) node[right] {$P=(1,1)$} ;
  \draw[gray, dotted] (P) -- (Pxy);
  \addplot3[domain=0:2*pi, samples=100, samples y=0, smooth, color2, dashed] ({sqrt(2)*cos(deg(x))}, {sqrt(2)*sin(deg(x))}, {2});
  \addplot3[domain=0:2*pi, samples=100, samples y=0, smooth, color2] ({sqrt(2)*cos(deg(x))}, {sqrt(2)*sin(deg(x))}, {0});
  \end{axis}
\end{tikzpicture}

\end{center}
\end{frame}



% ---------------------------------------------------------------------slide----
\begin{frame}
\frametitle{Level set of a scalar field}
\framesubtitle{Applications: Topographic maps and weather maps}
\begin{columns}
\begin{column}{0.5\textwidth}
\begin{center}
Topographic maps
\tikzsetnextfilename{derivatives_n_variables/mountain_level_curves}
\mode<presentation>{\resizebox{\textwidth}{!}{% Author: Alfredo Sánchez Alberca (asalber@ceu.es)
\begin{tikzpicture}
  \begin{axis}[
    %colormap name=twocolormap,
    3d box,
    width=7cm,
    view={25}{25},
    enlargelimits=false,
    grid=major,
    domain=-0.5:4.7,
    y domain=-2:2,
    samples=21,
    xlabel={Longitude},
    ylabel={Latitude},
    zlabel={Height},
    ]
    \addplot3 [y domain = 0:2, surf]
       {-0.7+4*exp(-0.5*(x+3))*(3*cos(4*x*180/pi)+2.5*cos(2*x*180/pi))
        + 0.5*y*y*4};
    \addplot3 [y domain = 0:2, contour gnuplot = {number=14, labels={false},
      draw color = black}, samples = 31, ]
      {-0.7+4*exp(-0.5*(x+3))*(3*cos(4*x*180/pi)+2.5*cos(2*x*180/pi))
       + 0.5*y*y*4};
    \addplot3 [contour gnuplot = {output point meta=rawz, number=14, labels={false},},
        samples=31,z filter/.code={\def\pgfmathresult{20}}]
        {-0.7+4*exp(-0.5*(x+3))*(3*cos(4*x*180/pi)+2.5*cos(2*x*180/pi))
         + 0.5*y*y*4};
    \addplot3 [y domain=-2:0,surf]
        {-0.7+4*exp(-0.5*(x+3))*(3*cos(4*x*180/pi)+2.5*cos(2*x*180/pi))
         + 0.5*y*y*4};
    \addplot3 [domain = 0:0.25, contour gnuplot = {number=14,labels={false},
        draw color=black}, samples=21]
        {-0.7+4*exp(-0.5*(x+3))*(3*cos(4*x*180/pi)+2.5*cos(2*x*180/pi))
         + 0.5*y*y*4};
  \end{axis}
\end{tikzpicture}}}
\mode<article>{\resizebox{0.5\textwidth}{!}{% Author: Alfredo Sánchez Alberca (asalber@ceu.es)
\begin{tikzpicture}
  \begin{axis}[
    %colormap name=twocolormap,
    3d box,
    width=7cm,
    view={25}{25},
    enlargelimits=false,
    grid=major,
    domain=-0.5:4.7,
    y domain=-2:2,
    samples=21,
    xlabel={Longitude},
    ylabel={Latitude},
    zlabel={Height},
    ]
    \addplot3 [y domain = 0:2, surf]
       {-0.7+4*exp(-0.5*(x+3))*(3*cos(4*x*180/pi)+2.5*cos(2*x*180/pi))
        + 0.5*y*y*4};
    \addplot3 [y domain = 0:2, contour gnuplot = {number=14, labels={false},
      draw color = black}, samples = 31, ]
      {-0.7+4*exp(-0.5*(x+3))*(3*cos(4*x*180/pi)+2.5*cos(2*x*180/pi))
       + 0.5*y*y*4};
    \addplot3 [contour gnuplot = {output point meta=rawz, number=14, labels={false},},
        samples=31,z filter/.code={\def\pgfmathresult{20}}]
        {-0.7+4*exp(-0.5*(x+3))*(3*cos(4*x*180/pi)+2.5*cos(2*x*180/pi))
         + 0.5*y*y*4};
    \addplot3 [y domain=-2:0,surf]
        {-0.7+4*exp(-0.5*(x+3))*(3*cos(4*x*180/pi)+2.5*cos(2*x*180/pi))
         + 0.5*y*y*4};
    \addplot3 [domain = 0:0.25, contour gnuplot = {number=14,labels={false},
        draw color=black}, samples=21]
        {-0.7+4*exp(-0.5*(x+3))*(3*cos(4*x*180/pi)+2.5*cos(2*x*180/pi))
         + 0.5*y*y*4};
  \end{axis}
\end{tikzpicture}}}
\end{center}
Level curves correspond to points with the height above the sea level.
\end{column}
\begin{column}{0.5\textwidth}
\begin{center}
Weather maps (Isobars)
\mode<presentation>{\includegraphics[width=\textwidth]{img/derivatives_n_variables/isobars}}
\mode<article>{\includegraphics[width=0.5\textwidth]{img/derivatives_n_variables/isobars}}
\end{center}
Level curves correspond to points with the same atmospheric pressure.
\end{column}
\end{columns}
\end{frame}


%---------------------------------------------------------------------slide----
\begin{frame}
\frametitle{Partial functions}
\begin{definition}[Partial function]
Given a scalar field $f:\mathbb{R}^n\rightarrow \mathbb{R}$, an $i$-th \emph{partial function} of $f$ is any function $f_i:\mathbb{R}\rightarrow \mathbb{R}$ that results of substituting all the variables of $f$ by constants, except the $i$-th variable, that is:
\[
f_i(x)=f(c_1,\ldots,c_{i-1},x,c_{i+1},\ldots,c_{n}),
\]
with $c_j$ $(j=1,\ldots, n,\ j\neq i)$ constants.
\end{definition}

\structure{\textbf{Example}}
If we take the function that measures the area of a triangle
\[f(x,y)=\frac{xy}{2},\]
and set the value of the base to $x=c$, then we the area of the triangle depends only of the height, and $f$ becomes a function of one variable, that is the partial function
\[
f_1(y)=f(c,y)=\frac{cy}{2},\quad \mbox{with $c$ constant}.
\]
\end{frame}



\subsection{Partial derivative notion}
%---------------------------------------------------------------------slide----
\begin{frame}
\frametitle{Variation of a function with respect to a variable}
We can measure the variation of a scalar field with respect to each of its variables in the same way that we measured the variation of a one-variable function.

Let $z=f(x,y)$ be a scalar field of $\mathbb{R}^2$.
If we are at point $(x_0,y_0)$ and we increase the value of $x$ a quantity $\Delta x$, then we move in the direction of the $x$-axis from the point $(x_0,y_0)$ to the point $(x_0+\Delta x,y_0)$, and the variation of the function is
\[
\Delta z=f(x_0+\Delta x,y_0)-f (x_0,y_0).
\]

Thus, the rate of change of the function with respect to $x$ along the interval $[x_0,x_0+\Delta x]$ is given by the quotient
\[\frac{\Delta z}{\Delta x}=\frac{f(x_0+\Delta x,y_0)-f(x_0,y_0)}{\Delta x}.\]
\end{frame}


%---------------------------------------------------------------------slide----
\begin{frame}
\frametitle{Instantaneous rate of change of a scalar field with respect to a variable}
If instead o measuring the rate of change in an interval, we measure the rate of change in a point, that is, when $\Delta x$ approaches 0, then we get the instantaneous rate of change:
\[
\lim_{\Delta x\rightarrow 0}\frac{\Delta z}{\Delta x}=\lim_{\Delta x \rightarrow 0}\frac{f(x_0+\Delta x,y_0)-f(x_0,y_0)}{\Delta x}.
\]
The value of this limit, if exists, it is known as the \emph{partial derivative} of $f$ with respect to the variable $x$ at the point $(x_0,y_0)$; it is written as
\[
\frac{\partial f}{\partial x}(x_0,y_0).
\]

This partial derivative measures the instantaneous rate of change of $f$ at the point $P=(x_0,y_0)$ when $P$ moves in the $x$-axis direction.
\end{frame}


%---------------------------------------------------------------------slide----
\begin{frame}
\frametitle{Geometric interpretation of partial derivatives}
Geometrically, a two-variable function $z=f(x,y)$ defines a surface.
If we cut this surface with a plane of equation $y=y_0$ (that is, the plane where $y$ is the constant $y_0$)
the intersection is a curve, and the partial derivative of $f$ with respect to to $x$ at $(x_0,y_0)$ is the slope of the tangent line to that curve at $x=x_0$.

\begin{center}
\tikzsetnextfilename{derivatives_n_variables/partial_tangent_surface}
%\href{https://ggbm.at/K3xnQRY8}{
\input{img/derivatives_n_variables/partial_tangent_surface}
%}
\end{center}
\end{frame}


%---------------------------------------------------------------------slide----
\begin{frame}
\frametitle{Partial derivative}
The concept of partial derivative can be extended easily from two-variable function to $n$-variables functions.

\begin{definition}[Partial derivative]
Given a $n$-variables function $f(x_1,\ldots,x_n)$, $f$ is \emph{partially differentiable} with respect to the variable $x_i$ at the point $a=(a_1,\ldots,a_n)$ if exists the limit
\[
\lim_{\Delta x_i\rightarrow 0} \frac{f(a_1,\ldots,a_{i-1},a_i+\Delta x_i,a_{i+1},\ldots,a_n)-f(a_1,\ldots,a_{i-1},a_i,a_{i+1},\ldots,a_n)} {h}.
\]
In such a case, the value of the limit is known as \emph{partial derivative} of $f$ with respect to $x_i$ at $a$; it is denoted
\[
f'_{x_i}(a)=\frac{\partial f}{\partial x_i}(a).
\]
\end{definition}

\structure{\bf Remark}
The definition of derivative for one-variable functions is a particular case of this definition for $n=1$.
\end{frame}


%---------------------------------------------------------------------slide----
\begin{frame}
\frametitle{Partial derivatives computation}
When we measure the variation of $f$ with respect to a variable $x_i$ at the point $a=(a_1,\ldots,a_n)$, the other variables remain constant.
Thus, if we can consider the $i$-th partial function
\[
f_i(x_i)=f(a_1,\ldots,a_{i-1},x_i,a_{i+1},\ldots,a_n),
\]

the partial derivative of $f$ with respect to $x_i$ can be computed differentiating this function:
\[
\frac{\partial f}{\partial x_i}(a)=f_i'(a_i).
\]

\begin{block}{Rule}
To differentiate partially $f(x_1,\ldots,x_n)$ with respect to the variable $x_i$, you have to differentiate $f$ as a function  of the variable $x_i$, considering the other variables as constants.
\end{block}
\end{frame}


%---------------------------------------------------------------------slide----
\begin{frame}
\frametitle{Partial derivatives computation}
\framesubtitle{Example of the volume of a perfect gas}
Consider the function that measures the volume of a perfect gas
\[v(t,p)=\frac{nRt}{p},\]
where $t$ is the temperature, $p$ the pressure and $n$ and $R$ are constants.

The instantaneous rate of change of the volume with respect to the pressure is the partial derivative of $v$ with respect to $p$.
To compute this derivative we have to think in $t$ as a constant and differentiate $v$ as if the unique variable was $p$:
\[
\frac{\partial v}{\partial p}(t,p)=\frac{d}{dp}\left(\frac{nRt}{p}\right)_{\mbox{\scriptsize $t=$cst}}=\frac{-nRt}{p^2}.
\]

In the same way, the instantaneous rate of change of the volume with respect to the temperature is the partial derivative of $v$ with respect to $t$:
\[
\frac{\partial v}{\partial t}(t,p)=\frac{d}{dt}\left(\frac{nRt}{p}\right)_{\mbox{\scriptsize$p=$cst}}=\frac{nR}{p}.
\]
\end{frame}



\subsection{Gradient}
%---------------------------------------------------------------------slide----
\begin{frame}
\frametitle{Gradient}
\begin{definition}[Gradient]
Given a scalar field $f(x_1,\ldots,x_n)$, the \emph{gradient} of $f$, denoted by $\nabla f$, is a function that maps every point
$a=(a_1,\ldots,a_n)$ to a vector with coordinates the partial derivatives of $f$ at $a$,
\[
\nabla f(a)=\left(\frac{\partial f}{\partial x_1}(a),\ldots,\frac{\partial f}{\partial x_n}(a)\right).
\]
\end{definition}

Later we will show that the gradient in a point is a vector with the magnitude and direction of the maximum rate of change of the function in that point.
Thus, \alert{\emph{$\nabla f(a)$ shows the direction of maximum increase of $f$ at $a$}}, while $-\nabla f(a)$ show the direction of maximum decrease of $f$ at $a$.
\end{frame}



% ---------------------------------------------------------------------slide----
\begin{frame}
\frametitle{Gradient computation}
\framesubtitle{Example with a temperature function}

After heating a surface, the temperature $t$ (in $^\circ$C) at each point $(x,y,z)$ (in m) of the surface is given by the function
\[
t(x,y,z)=\frac{x}{y}+z^2.
\]
\emph{In what direction will increase the temperature faster at point $(2,1,1)$ of the surface?
What magnitude will the maximum increase of temperature have?}

The direction of maximum increase of the temperature is given by the gradient
\[
\nabla t(x,y,z)=\left(\frac{\partial t}{\partial x}(x,y,z),\frac{\partial t}{\partial y}(x,y,z),\frac{\partial t}{\partial
z}(x,y,z)\right)=\left(\frac{1}{y},\frac{-x}{y^2},2z\right).
\]

At point $(2,1,1)$ de direction is given by the vector
\[
\nabla t(2,1,1)=\left(\frac{1}{1},\frac{-2}{1^2},2\cdot 1\right)=(1,-2,2),
\]
and its magnitude is
\[
|\nabla f(2,1,1)|=|\sqrt{1^2+(-2)^2+2^2}|=|\sqrt{9}|=3 \mbox{ $^\circ$C/m}.
\]
\end{frame}



\subsection{Composition of a vectorial field with a scalar field}
%---------------------------------------------------------------------slide----
\begin{frame}
\frametitle{Composition of a vectorial field with a scalar field\\ Multivariate chain rule}
If $f:\mathbb{R}^n\rightarrow \mathbb{R}$ is a scalar field and $g:\mathbb{R}\rightarrow \mathbb{R}^n$ is a vectorial function, then it is possible to compound $g$ with $f$, so that $f\circ g:\mathbb{R}\rightarrow \mathbb{R}$ is a one-variable function.

\begin{theorem}[Chain rule]
If $g(t)=(x_1(t),\ldots,x_n(t))$ is a vectorial function differentiable at $t$ and $f(x_1,\ldots,x_n)$ is a scalar field differentiable at the point $g(t)$, then $f\circ g(t)$ is differentiable at $t$ and
\[
(f\circ g)'(t) = \nabla f(g(t))\cdot g'(t)=\frac{\partial f}{\partial x_1}\frac{dx_1}{dt}+ \cdots + \frac{\partial f}{\partial x_n}\frac{dx_n}{dt}
\]
\end{theorem}
\end{frame}


%---------------------------------------------------------------------slide----
\begin{frame}
\frametitle{Composition of a vectorial field with a scalar field\\ Multivariate chain rule}
\framesubtitle{Example}
Let us consider the scalar field $f(x,y)=x^2y$ and the vectorial function $g(t)=(\cos t,\sin t)$ $t\in [0,2\pi]$ in the real plane, then
\[
\nabla f(x,y) = (2xy, x^2) \quad  \mbox{and} \quad g'(t) = (-\sin t, \cos t),
\]
and
\begin{align*}
(f\circ g)'(t) &= \nabla f(g(t))\cdot g'(t) = (2\cos t\sin t,\cos^2 t)\cdot (-\sin t,\cos t) =\\
&= -2\cos t\sin^2 t+\cos^3 t.
\end{align*}

We can get the same result differentiating the composed function directly
\[
(f\circ g)(t) = f(g(t)) = f(\cos t, \sin t) = \cos^2 t\sin t,
\]
and its derivative is
\[
(f\circ g)'(t) = 2\cos t(-\sin t)\sin t+\cos^2 t \cos t = -2\cos t\sin^2 t+\cos^3 t.
\]
\end{frame}


%---------------------------------------------------------------------slide----
\begin{frame}
\frametitle{Multivariate chain rule}
The chain rule for the composition of a vectorial function with a scalar field allow us to get the algebra of derivatives for one-variable functions easily:
\begin{align*}
(u+v)' &= u'+v'\\
(uv)' &= u'v+uv'\\
\left(\frac{u}{v}\right)' &= \frac{u'v-uv'}{v^2}\\
(u\circ v)' &= u'(v)v'
\end{align*}

To infer the derivative of the sum of two functions $u$ and $v$, we can take the scalar field $f(x,y)=x+y$ and the vectorial function $g(t)=(u(t),v(t))$.
Applying the chain rule we get
\[
(u+v)'(t) = (f\circ g)'(t) = \nabla f(g(t))\cdot g'(t) = (1,1)\cdot (u',v') = u'+v'.
\]
To infer the derivative of the quotient of two functions $u$ and $v$, we can take the scalar field $f(x,y)=x/y$ and the vectorial function $g(t)=(u(t),v(t))$.
\[
\left(\frac{u}{v}\right)'(t) = (f\circ g)'(t) = \nabla f(g(t))\cdot g'(t) = \left(\frac{1}{v},-\frac{u}{v^2}\right)\cdot (u',v') = \frac{u'v-uv'}{v^2}.
\]
\end{frame}



\subsection{Directional derivative}
%---------------------------------------------------------------------slide----
\begin{frame}
\frametitle{Directional derivative}
For a scalar field $f(x,y)$, we have seen that the partial derivative $\dfrac{\partial f}{\partial x}(x_0,y_0)$ is the instantaneous rate of change of $f$ with respect to $x$ at point $P=(x_0,y_0)$, that is, when we move along the $x$-axis.

In the same way, $\dfrac{\partial f}{\partial y}(x_0,y_0)$ is the instantaneous rate of change of $f$ with respect to $y$ at the point $P=(x_0,y_0)$, that is, when we move along the $y$-axis.

But, \emph {what happens if we move along any other direction?}

The instantaneous rate of change of $f$ at the point $P=(x_0,y_0)$ along the direction of a unitary vector $u$ is known as \emph{directional derivative}.
\end{frame}


%---------------------------------------------------------------------slide----
\begin{frame}
\frametitle{Directional derivative}
\begin{definition}[Directional derivative]
Given a scalar field $f$ of $\mathbb{R}^n$, a point $P$ and a unitary vector $\mathbf{u}$ in that space, we say that $f$ is differentiable at $P$ along the direction of $\mathbf{u}$ if exists the limit
\[
f'_{\mathbf{u}}(P) = \lim_{h\rightarrow 0}\frac{f(P+h\mathbf{u})-f(P)}{h}.
\]
In such a case, the value of the limit is known as \emph{directional derivative} of $f$ at the point $P$ along the direction of $\mathbf{u}$.
\end{definition}

If we consider a unitary vector $\mathbf{u}$, the trajectory that passes through $P$, following the direction of $\mathbf{u}$, has equation
\[
g(t)=P+t\mathbf{u},\ t\in\mathbb{R}.
\]

For $t=0$, this trajectory passes through the point $P=g(0)$ with velocity $\mathbf{u}=g'(0)$.

Thus, the directional derivative of $f$ at the point $P$ along the direction of $\mathbf{u}$ is
\[
(f\circ g)'(0) = \nabla f(g(0))\cdot g'(0) = \nabla f(P)\cdot \mathbf{u}.
\]

\textbf{Remark}: The partial derivatives are the directional derivatives along the vectors of the canonical basis.
\end{frame}


%---------------------------------------------------------------------slide----
\begin{frame}
\frametitle{Directional derivative}
\framesubtitle{Example}
Given the function $f(x,y) = x^2+y^2$, its gradient is
\[
\nabla f(x,y) = (2x,2y).
\]

The directional derivative of $f$ at the point $P=(1,1)$, along the unit vector $\mathbf{u}=(1/\sqrt{2},1/\sqrt{2})$ is
\[
f'_{\mathbf{u}}(P) = \nabla f(P)\cdot \mathbf{u} = (2,2)\cdot(1/\sqrt{2},1/\sqrt{2}) = \frac{2}{\sqrt{2}}+\frac{2}{\sqrt{2}} = \frac{4}{\sqrt{2}}.
\]

To compute the directional derivative along a non-unitary vector $\mathbf{v}$, we have to use the unitary vector that results from normalizing $v$ with the transformation:
\[
\mathbf{v'}=\frac{\mathbf{v}}{|\mathbf{v}|}.
\]
\end{frame}


%---------------------------------------------------------------------slide----
\begin{frame}
\frametitle{Geometric interpretation of the directional derivative}
Geometrically, a two-variable function $z=f(x,y)$ defines a surface.
If we cut this surface with a plane of equation $a(y-y_0)=b(x-x_0)$ (that is, the vertical plane that passes through the point $P=(x_0,y_0)$ with the direction of vector $\mathbf{u}=(a,b)$)
the intersection is a curve, and the directional derivative of $f$ at $P$ along the direction of $\mathbf{u}$ is the slope of the tangent line to that curve at point $P$.

\begin{center}
\tikzsetnextfilename{derivatives_n_variables/directional_tangent_surface}
%\href{https://ggbm.at/Bx8nFMNc}{
% Author: Alfredo Sánchez Alberca (asalber@ceu.es)
\begin{tikzpicture}
  \begin{axis}[view={120}{20},
  gen3dfun,
  xmin=-1.5, xmax=1.5,
  ymin=-1.5, ymax=1.5,
  zmin=-0.2, zmax=4,
  axis x line=middle,
  axis y line=middle,
  axis z line=middle,
%  axis equal=true,
  every axis x label/.style={at={(xticklabel* cs:1.1)}},
  every axis y label/.style={at={(yticklabel* cs:1.1)}},
  every axis z label/.style={at={(zticklabel* cs:1.1)}},
  x tick label style={anchor=east, inner sep=2pt}, 
  y tick label style={anchor=north, inner sep=5pt},
  z tick label style={anchor=east, inner sep=2pt},  
  xtick={1},
  xticklabels={$x_0$},
  ytick={1},
  yticklabels={$y_0$}, 
  ztick={2},
  zticklabels={}{$f(x_0,y_0)$},
  mesh/interior colormap name=incolormap,
  colormap name=outcolormap,
  clip=false,
  height=5cm,
  width=5cm,
  ]
  \addplot3[surf, domain=-1.5:1.5, y domain=-1.5:1, faceted color=color1, samples=30, samples y=30, shader=interp] {x^2+y^2/2};
  \coordinate (O) at (0,0,0);
  \coordinate (P) at (1,1,1.5);
  \coordinate (Pxy) at (1,1,0);
  \addplot3[gray, fill=gray, opacity=0.8] coordinates{(0,0,0) (2,2,0) (2,2,5) (0,0,5)};
%  \addplot3[surf, domain=-1.5:1.5, y domain=1:1.5, faceted color=color1, samples=30, samples y= 5, shader=interp] {x^2+y^2/2};
  \fill (P) circle (1.2pt) ;
  \addplot3[domain=0.5:1.8, samples y=0] ({x},{x},{x^2+x^2/2});
  \draw[color2, dotted] (P) -- (Pxy);
  \draw[color2, dotted] (Pxy) -- (1,0,0);
  \draw[color2, dotted] (Pxy) -- (0,1,0);
  \draw[color2, dotted] (P) -- (0,0,2);
  \end{axis}
\end{tikzpicture}

%}
\end{center}
\end{frame}



%---------------------------------------------------------------------slide----
\begin{frame}
\frametitle{Growth of scalar field along the gradient}
We have seen that for any vector $\mathbf{u}$
\[
f'_{\mathbf{u}}(P) = \nabla f(P)\cdot \mathbf{u} = |\nabla f(P)|\cos \theta,
\]
where $\theta$ is the angle between $\mathbf{u}$ and the gradient $\nabla f(P)$.

Taking into account that $-1\leq \cos\theta\leq 1$, for any vector $\mathbf{u}$ it is satisfied that
\[
-|\nabla f(P)|\leq f'_{\mathbf{u}}(P)\leq |\nabla f(P)|.
\]
Furthermore, if $\mathbf{u}$ has the same direction and sense than the gradient, we have $f'_{\mathbf{u}}(P)=|\nabla f(P)|\cos 0=|\nabla f(P)|$.
Therefore, \alert{\emph{the maximum increase of a scalar field at a point $P$ is along the direction of the gradient at that point}}.

In the same manner, if $\mathbf{u}$ has the same direction but opposite sense than the gradient, we have $f'_{\mathbf{u}}(P)=|\nabla f(P)|\cos \pi=-|\nabla f(P)|$.
Therefore, \alert{\emph{the maximum decrease of a scalar field at a point $P$ is along the opposite direction of the gradient at that point}}.
\end{frame}
%
%
\subsection{Implicit derivation}
%---------------------------------------------------------------------slide----
\begin{frame}
\frametitle{Implicit functions}
When we have a relation $f(x,y)=0$, sometimes we can consider $y$ as an \emph{implicit function} of $x$, at least in a neighborhood of a point $(x_0,y_0)$.

\structure{\textbf{Example}}
The equation $x^2+y^2=25$, whose graph is the circle of radius 5 centered at the origin of coordinates, its not a function, because if we solve the equation for $y$, we have two images for some values of $x$,
\[
y=\pm \sqrt{25-x^2}
\]

However, near the point $(3,4)$ we can represent the relation as the function $y=\sqrt{25-x^2}$, and near the point $(3,-4)$ we can represent the relation as the function $y=-\sqrt{25-x^2}$.
\end{frame}


%---------------------------------------------------------------------slide----
\begin{frame}
\frametitle{Implicit derivation}
If an equation $f(x,y)=0$ defines $y$ as a implicit function of $x$, $y=h(x)$, in a neighborhood of $(x_0,y_0)$, then
we can compute de derivative of $y$, $h'(x)$, even if we do not know the explicit formula for $h$.

\begin{theorem}[Implicit function (one-variable)]
Let $f(x,y):\mathbb{R}^2\longrightarrow \mathbb{R}$ a two-variables function and let $(x_0,y_0)$ be a point in $\mathbb{R}^2$ such that $f(x_0,y_0)=0$.
If $f$ has partial derivatives continuous at $(x_0,y_0)$ and $\frac{\partial f}{\partial y}(x_0,y_0)\neq 0$, then there is an open interval $I\subset \mathbb{R}$ with $x_0\in I$ and a function $h(x): I\longrightarrow \mathbb{R}$ such that
\begin{enumerate}
\item $y_0=h(x_0)$.
\item $f(x,h(x))=0$ for all $x\in I$.
\item $h$ is differentiable on $I$, and $y'=h'(x)=\frac{-\dfrac{\partial f}{\partial x}}{\dfrac{\partial f}{\partial y}}$
\end{enumerate}
\end{theorem}
\end{frame}


%---------------------------------------------------------------------slide----
\begin{frame}
\frametitle{Implicit derivation}
\framesubtitle{Proof}
To prove the last result, take the trajectory $g(x)=(x,h(x))$ on the interval $I$.
Then
\[
(f\circ g)(x) = f(g(x)) = f(x,h(x))=0.
\]

Thus, using the chain rule we have

\[
(f\circ g)'(x) = \nabla f(g(x))\cdot g'(x) = \left(\frac{\partial f}{\partial x}, \frac{\partial f}{\partial y}\right)\cdot (1,h'(x)) =
\frac{\partial f}{\partial x}+\frac{\partial f}{\partial y}h'(x) = 0,
\]
from where we can deduce
\[
y'=h'(x)=\frac{-\dfrac{\partial f}{\partial x}}{\dfrac{\partial f}{\partial y}}
\]

This technique that allows us to compute $y'$ in a neighborhood of $x_0$ without the explicit formula of $y=h(x)$, it is known as \emph{implicit derivation}.
\end{frame}


%---------------------------------------------------------------------slide----
\begin{frame}
\frametitle{Implicit derivation}
\framesubtitle{Example}
Consider the equation of the circle of radius 5 centered at the origin $x^2+y^2=25$.
It can also be written as
\[
f(x,y) = x^2+y^2-25 = 0.
\]
Take the point $(3,4)$ that satisfies the equation, $f(3,4)=0$.

As $f$ have partial derivatives $\frac{\partial f}{\partial x}=2x$ and $\frac{\partial f}{\partial y}=2y$, that are continuous at $(3,4)$, and $\frac{\partial f}{\partial y}(3,4)=8\neq 0$, then $y$ can be expressed as a function of $x$ in a neighborhood of $(3,4)$ and its derivative is
\[
y'=\frac{-\frac{\partial f}{\partial x}}{\frac{\partial f}{\partial y}} = \frac{-2x}{2y}=\frac{-x}{y} \quad and \quad y'(3)=\frac{-3}{4}.
\]

In this particular case, that we know the explicit formula of $y=\sqrt{1-x^2}$, we can get the same result computing the derivative as usual
\[
y' = \frac{1}{2\sqrt{1-x^2}}(-2x) = \frac{-x}{\sqrt{1-x^2}}.
\]
\end{frame}


%---------------------------------------------------------------------slide----
\begin{frame}
\frametitle{Implicit derivation}
The implicit function theorem can be generalized to functions with several variables.

\begin{theorem}[Implicit function]
Let $f(x_1,\ldots,x_n,y):\mathbb{R}^{n+1}\longrightarrow \mathbb{R}$ a $n+1$-variables function and let $(x_1^0,\ldots, x_2^0,y^0)$ be a point in $\mathbb{R}^{n+1}$ such that $f(x_1^0,\ldots,x_n^0,y^0)=0$.
If $f$ has partial derivatives continuous at $(x_1^0,\ldots,x_n^0,y^0)$ and $\frac{\partial f}{\partial y}(x_1^0,\ldots,x_n^0,y^0)\neq 0$, then there is a region $I\subset \mathbb{R}^n$ with $(x_1^0,\ldots,x_n^0)\in I$ and a function $h(x1,\ldots, xn): I\longrightarrow \mathbb{R}$ such that
\begin{enumerate}
\item $y_0=h(x_1^0,\ldots,x_n^0)$.
\item $f(x_1,\ldots,x_n,h(x_1,\ldots,x_n))=0$ for all $(x_1,\ldots,x_n)\in I$.
\item $h$ is differentiable on $I$, and $\dfrac{\partial y}{\partial x_i}=\frac{-\dfrac{\partial f}{\partial x_i}}{\dfrac{\partial f}{\partial y}}$
\end{enumerate}
\end{theorem}
\end{frame}


%---------------------------------------------------------------------slide----
\begin{frame}
\frametitle{Gradient property}
\begin{theorem}
Let $C$ be the level set of a scalar field $f$ that includes a point $P$.
If $\mathbf{v}$ is the velocity at $P$ of a trajectory following $C$, then
\[
\nabla f(P) \cdot \mathbf{v} = 0.
\]
that is, the gradient of $f$ at $P$ is normal to $C$ at $P$, provided that the gradient is not zero.
\end{theorem}

\structure{\textbf{Proof}}
If we take the trajectory $g(t)$ that follows the level set $C$ and passes through $P$ at time $t=t_0$, that is $P=g(t_0)$, so $\mathbf{v}=g'(t_0)$, then
\[
(f\circ g)(t) = f(g(t)) = f(P),
\]
that is constant at any $t$.
Thus, applying the chain rule we have
\[
(f\circ g)'(t) = \nabla f(g(t))\cdot  g'(t) = 0,
\]
and, particularly, at $t=t_0$, we have
\[
\nabla f(P)\cdot \mathbf{v} = 0.
\]
\end{frame}


%---------------------------------------------------------------------slide----
\begin{frame}
\frametitle{Normal and tangent line to curve in the plane}
According to the previous result, the normal line to a curve with equation $f(x,y)=0$ at point $P=(x_0,y_0)$, has equation
\[
P+t\nabla f(P) = (x_0,y_0)+t\nabla f(x_0,y_0).
\]
\structure{\textbf{Example}}
Given the scalar field $f(x,y)=x^2+y^2-25$, and the point $P=(3,4)$, the level set of $f$ that passes through $P$, that satisfies $f(x,y)=f(P)=0$, is the circle with radius 5 centered at the origin of coordinates.
Thus, taking as a normal vector the gradient of $f$
\[
\nabla f(x,y) = (2x,2y),
\]
at the point $P=(3,4)$ is $\nabla f(3,4) = (6,8)$, and the normal line to the circle at $P$ is
\[
P+t\nabla f(P) = (3,4)+t(6,8) = (3+6t,4+8t),
\]

On the other hand, the tangent line to the circle at $P$ is
\[
((x,y)-P)\cdot \nabla f(P) = ((x,y)-(3,4))\cdot (6,8) = (x-3,y-4)\cdot(6,8) = 6x+8y=50.
\]
\end{frame}


%---------------------------------------------------------------------slide----
\begin{frame}
\frametitle{Normal line and tangent plane to a surface in the space}
In the same way, if we have a surface with equation $f(x,y,z)=0$, at the point $P=(x_0,y_0,z_0)$ the normal line has equation
\[
P+t\nabla f(P) = (x_0,y_0,z_0)+t\nabla f(x_0,y_0,z_0).
\]
\structure{\textbf{Example}}
Given the scalar field $f(x,y,z)=x^2+y^2-z$, and the point $P=(1,1,2)$, the level set of $f$ that passes through $P$, that satisfies $f(x,y)=f(P)=0$, is the paraboloid $z=x^2+y^2$.
Thus, taking as a normal vector the gradient of $f$
\[
\nabla f(x,y,z) = (2x,2y,-1),
\]
at the point $P=(1,1,2)$ is $\nabla f(1,1,2) = (2,2,-1)$, and the normal line to the paraboloid at $P$ is
\[
P+t\nabla f(P) = (1,1,2)+t\nabla f(1,1,2) = (1,1,2)+t(2,2,-1) = (1+2t,1+2t,2-t).
\]

On the other hand, the tangent plane to the paraboloid at $P$ is
\begin{align*}
((x,y,z)-P)\cdot \nabla f(P) &= ((x,y,z)-(1,1,2))(2,2,-1) = (x-1,y-1,z-2)(2,2,-1)=\\
 &= 2(x-1)+2(y-1)-(z-2) = 2x+2y-z-2= 0.
\end{align*}
\end{frame}

%
% %---------------------------------------------------------------------slide----
% \begin{frame}
% \frametitle{Recta normal y plano tangente a una superficie}
% \framesubtitle{Example}
% La gráfica del paraboloide $f(x,y,z)=x^2+y^2-z=0$ y su plano tangente en el punto $P=(1,1,2)$ es
% \begin{center}
% \scalebox{1}{\input{img/calculo_diferencial_n_variables/plano_tangente_paraboloide}}
% \end{center}
% \end{frame}


\subsection{Second order partial derivatives}
%---------------------------------------------------------------------slide----
\begin{frame}
\frametitle{Second order partial derivatives}
As the partial derivatives of a function are also functions of several variables we can differentiate partially each of them.

If a function $f(x_1,\ldots,x_n)$ has a partial derivative $f'_{x_i}(x_1,\ldots,x_n)$ with respect to the variable $x_i$ in a set $A$, then we can differentiate partially again $f'_{x_i}$ with respect to the variable $x_j$.
This second derivative, when exists, is known as \emph{second order partial derivative} of $f$ with respect to the variables $x_i$ and $x_j$; it is written as
\[
\frac{\partial ^2 f}{\partial x_j \partial x_i}= \frac{\partial}{\partial x_j}\left(\frac{\partial f}{\partial x_i}\right).
\]

In the same way we can define higher order partial derivatives.
\end{frame}


%---------------------------------------------------------------------slide----
\begin{frame}
\frametitle{Second order partial derivatives computation}
\framesubtitle{Example}
The two-variables function
\[f(x,y)=x^y\]
has 4 second order partial derivatives:
\begin{align*}
\frac{\partial^2 f}{\partial x^2}(x,y) &=
\frac{\partial}{\partial x}\left(\frac{\partial f}{\partial x}(x,y)\right) =
\frac{\partial}{\partial x}\left(yx^{y-1}\right) =
y(y-1)x^{y-2},\\
\frac{\partial^2 f}{\partial y \partial x}(x,y) &=
\frac{\partial}{\partial y}\left(\frac{\partial f}{\partial x}(x,y)\right) =
\frac{\partial}{\partial y}\left(yx^{y-1}\right) =
x^{y-1}+yx^{y-1}\log x,\\
\frac{\partial^2 f}{\partial x \partial y}(x,y) &=
\frac{\partial}{\partial x}\left(\frac{\partial f}{\partial y}(x,y)\right) =
\frac{\partial}{\partial x}\left(x^y\log x \right) =
yx^{y-1}\log x+x^y\frac{1}{x},\\
\frac{\partial^2 f}{\partial y^2}(x,y) &=
\frac{\partial}{\partial y}\left(\frac{\partial f}{\partial y}(x,y)\right) =
\frac{\partial}{\partial y}\left(x^y\log x \right) =
x^y(\log x)^2.
\end{align*}
\end{frame}



\subsection{Hessian matrix}
%---------------------------------------------------------------------slide----
\begin{frame}
\frametitle{Hessian matrix and Hessian}
\begin{definition}[Hessian matrix]
Given a scalar field $f(x_1,\ldots,x_n)$, with second order partial derivatives at the point $a=(a_1,\ldots,a_n)$, the \emph{Hessian matrix} of $f$ at $a$, denoted by $\nabla^2f(a)$, is the matrix
\[
\nabla^2f(a)=\left(
\begin{array}{cccc}
\dfrac{\partial^2 f}{\partial x_1^2}(a) &
\dfrac{\partial^2 f}{\partial x_1 \partial x_2}(a) &
\cdots &
\dfrac{\partial^2 f}{\partial x_1 \partial x_n}(a)\\
\dfrac{\partial^2 f}{\partial x_2 \partial x_1}(a) &
\dfrac{\partial^2 f}{\partial x_2^2}(a) &
\cdots &
\dfrac{\partial^2 f}{\partial x_2 \partial x_n}(a)\\
\vdots & \vdots & \ddots & \vdots \\
\dfrac{\partial^2 f}{\partial x_n \partial x_1}(a) &
\dfrac{\partial^2 f}{\partial x_n \partial x_2}(a) &
\cdots &
\dfrac{\partial^2 f}{\partial x_n^2}(a)
\end{array}
\right)
\]
The determinant of this matrix is known as \emph{Hessian} of $f$ at $a$; it is denoted $Hf(a)=|\nabla^2f(a)|$.
\end{definition}
\end{frame}


%---------------------------------------------------------------------slide----
\begin{frame}
\frametitle{Hessian matrix computation}
Consider again the two-variables function
\[f(x,y)=x^y.\]
Its Hessian matrix is
\[
\nabla^2f(x,y)=\left(
\begin{array}{cc}
\dfrac{\partial^2 f}{\partial x^2} & \dfrac{\partial^2 f}{\partial x \partial y}\\
\dfrac{\partial^2 f}{\partial y \partial x} & \dfrac{\partial^2 f}{\partial y^2}
\end{array}
\right)
=
\left(
\begin{array}{cc}
y(y-1)x^{y-2} & x^{y-1}(y\log x+1) \\
x^{y-1}(y\log x+1) & x^y(\log x)^2
\end{array}
\right).
\]

At point $(1,2)$ is
\[
\nabla^2f(1,2)=\left(
\begin{array}{cc}
2(2-1)1^{2-2} & 1^{2-1}(2\log 1+1) \\
1^{2-1}(2\log 1+1) & 1^2(\log 1)^2
\end{array}
\right)
=
\left(
\begin{array}{cc}
2 & 1 \\
1 & 0
\end{array}
\right).
\]

And its Hessian is
\[
Hf(1,2)=\left|
\begin{array}{cc}
2 & 1 \\
1 & 0
\end{array}
\right|=
2\cdot 0-1\cdot1= -1.
\]
\end{frame}


%---------------------------------------------------------------------slide----
\begin{frame}
\frametitle{Igualdad de las derivatives cruzadas}
En el ejemplo anterior se aprecia que las \emph{derivatives cruzadas} de segundo orden $\frac{\partial^2 f}{\partial y\partial x}$ y $\frac{\partial^2 f}{\partial x\partial y}$ coinciden. Ello es debido al siguiente theorem:

\begin{theorem}[Igualdad derivatives cruzadas]
Si $f(x,y)$ es una function tal que sus derivatives parciales $\frac{\partial f}{\partial x}$, $\frac{\partial f}{\partial y}$, $\frac{\partial^2 f}{\partial y\partial x}$ y $\frac{\partial^2 f}{\partial x\partial y}$ existen y son continuas en un conjunto abierto $A$, entonces
\[
\frac{\partial^2 f}{\partial y\partial x}=\frac{\partial^2 f}{\partial x\partial y}.
\]
\end{theorem}

Una consecuencia del theorem es que, al calcular una derivative parcial de segundo orden que cumpla lo anterior, \alert{\emph{¡el orden en que se realicen las derivatives parciales no importa!}}

Si el theorem se cumple para todas las derivatives parciales de segundo orden, entonces la matriz hessiana es simétrica.
\end{frame}
% 
% 
% 
% \subsection{Fórmula de Taylor}
% %---------------------------------------------------------------------slide----
% \begin{frame}
% \frametitle{Aproximación lineal de un campo escalar}
% Ya se vio cómo aproximar functions de una variable mediante polinomios de Taylor. 
% Esto también se puede generalizar a la aproximación de campos escalares mediante polinomios de varias variables. 
% 
% Si $P$ es un punto del dominio de un campo escalar $f$ y $\mathbf{v}$ un vector, la \emph{fórmula de Taylor} de primer grado de $f$ alrededor del punto $P$ es
% \[
% f(P+\mathbf{v}) = f(P) + \nabla f(P)\cdot \mathbf{v} +R^1_{f,P}(\mathbf{v}),
% \]
% donde 
% \begin{align*}
% P^1_{f,P}(\mathbf{v}) = f(P)+\nabla f(P)\mathbf{v}
% \end{align*}
% es el \emph{polinomio de Taylor} de primer grado de $f$ en el punto $P$, y $R^1_{f,P}(\mathbf{v})$ es el \emph{resto de taylor} para el vector $\mathbf{v}$, y mide el error cometido en la aproximación.
% 
% Se cumple que  
% \[
% \lim_{|\mathbf{v}|\rightarrow 0} \frac{R^1_{f,P}(\mathbf{v})}{|\mathbf{v}|} = 0
% \]
% 
% Obsérvese que el polinomio de Taylor de primer grado coincide con el plano tangente a $f$ en $P$.
% \end{frame}
% 
% 
% %---------------------------------------------------------------------slide----
% \begin{frame}
% \frametitle{Aproximación lineal de un campo escalar}
% \framesubtitle{Campo escalar de dos variables}
% Si $f$ es un campo escalar de dos variables $f(x,y)$ y $P=(x_0,y_0)$, teniendo en cuenta que para un punto cualquiera $Q=(x,y)$, el vector $\mathbf{v}=\vec{PQ}=(x-x_0,y-y_0)$, el polinomio de Taylor de $f$
% en el punto $P$, puede expresarse
% \begin{align*}
% P^1_{f,P}(x,y) &= f(x_0,y_0)+\nabla f(x_0,y_0)(x-x_0,y-y_0) =\\
% &= f(x_0,y_0)+\frac{\partial f}{\partial x}(x_0,y_0)(x-x_0)+\frac{\partial f}{\partial y}(x_0,y_0)(y-y_0).
% \end{align*}
% \end{frame}
% 
% 
% %---------------------------------------------------------------------slide----
% \begin{frame}
% \frametitle{Aproximación lineal de un campo escalar}
% \framesubtitle{Example}
% Dado el campo escalar $f(x,y)=\log(xy)$, su gradiente es
% \[
% \nabla f(x,y) = \left(\frac{1}{x},\frac{1}{y}\right),
% \]
% y el polinomio de Taylor de primer grado en el punto $P=(1,1)$ es
% \begin{align*}
% P^1_{f,P}(x,y) &= f(1,1) +\nabla f(1,1)\cdot (x-1,y-1) = \\
% &= \log 1+(1,1)\cdot(x-1,y-1) = x-1+y-1 = x+y-2.\\
% \end{align*}
% Este polinomio, permite aproximar el valor de $f$ cerca del punto $P$.
% Por ejemplo
% \[ 
% f(1.01,1.01) \approx P^1_{f,P}(1.01,1.01) = 1.01+1.01-2 = 0.02.
% \]
% \end{frame}
% 
% 
% %---------------------------------------------------------------------slide----
% \begin{frame}
% \frametitle{Aproximación cuadrática de un campo escalar}
% Si $P$ es un punto del dominio de un campo escalar $f$ y $\mathbf{v}$ un vector, la \emph{fórmula de Taylor} de segundo
% grado de $f$ alrededor del punto $P$ es
% \[
% f(P+\mathbf{v}) = f(P) + \nabla f(P)\cdot \mathbf{v} + \frac{1}{2}\nabla^2f(P)\mathbf{v}\cdot\mathbf{v} + R^2_{f,P}(\mathbf{v}),
% \]
% donde 
% \begin{align*}
% P^2_{f,P}(\mathbf{v})&=f(P)+\nabla f(P)\mathbf{v}+\frac{1}{2}\nabla^2f(P)\mathbf{v}\cdot\mathbf{v}
% \end{align*}
% es el \emph{polinomio de Taylor} de segundo grado de $f$ en el punto $P$, y $R^2_{f,P}(\mathbf{v})$ es el \emph{resto de taylor} para el vector $\mathbf{v}$.
% 
% Se cumple que  
% \[
% \lim_{|\mathbf{v}\rightarrow 0|} \frac{R^2_{f,P}(\mathbf{v})}{|\mathbf{v}|^2} = 0
% \]
% lo que indica que el resto es mucho más pequeño que el cuadrado del módulo de $\mathbf{v}$.
% \end{frame}
% 
% 
% %---------------------------------------------------------------------slide----
% \begin{frame}
% \frametitle{Aproximación cuadrática de un campo escalar}
% \framesubtitle{Campo escalar de dos variables}
% Si $f$ es un campo escalar de dos variables $f(x,y)$ y $P=(x_0,y_0)$, el polinomio de Taylor de $f$ en el punto $P$, puede expresarse
% \begin{multline*}
% P^2_{f,P}(x,y) = f(x_0,y_0)+\nabla f(x_0,y_0)(x-x_0,y-y_0) +\\
% +\frac{1}{2}(x-x_0,y-y_0)\nabla^2f(x_0,y_0)(x-x_0,y-y_0)= \\
% = f(x_0,y_0)+\frac{\partial f}{\partial x}(x_0,y_0)(x-x_0)+\frac{\partial f}{\partial y}(x_0,y_0)(y-y_0)+\\
% +\frac{1}{2}\left(\frac{\partial^2 f}{\partial x^2}(x_0,y_0) (x-x_0)^2 + 2\frac{\partial^2 f}{\partial y\partial x}(x_0,y_0) (x-x_0)(y-y_0) + \frac{\partial^2 f}{\partial y^2}(x_0,y_0) (y-y_0)^2\right)
% \end{multline*}
% \end{frame}
% 
% 
% %---------------------------------------------------------------------slide----
% \begin{frame}
% \frametitle{Aproximación cuadrática de un campo escalar}
% \framesubtitle{Example}
% Dado el campo escalar $f(x,y)=\log(xy)$, su gradiente es
% \[
% \nabla f(x,y) = \left(\frac{1}{x},\frac{1}{y}\right),
% \]
% y su matriz hessiana es 
% \[
% Hf(x,y) = \left(
% \begin{array}{cc}
% \frac{-1}{x^2} & 0\\
% 0 & \frac{-1}{y^2}
% \end{array}
% \right)
% \]
% y el polinomio de Taylor de segundo grado en el punto $P=(1,1)$ es
% \begin{align*}
% P^2_{f,P}(x,y) &= f(1,1) +\nabla f(1,1)\cdot (x-1,y-1) + \frac{1}{2}(x-1,y-1)\nabla^2f(1,1)\cdot(x-1,y-1)=\\
% &= \log 1+(1,1)\cdot(x-1,y-1) + \frac{1}{2}(x-1,y-1)
% \left(
% \begin{array}{cc}
% -1 & 0\\
% 0 & -1
% \end{array}
% \right)
% \left(
% \begin{array}{c}
% x-1\\
% y-1
% \end{array}
% \right)
% = \\
% &= x-1+y-1+\frac{-x^2-y^2+2x+2y-2}{2} = \frac{-x^2-y^2+4x+4y-6}{2}.
% \end{align*}
% Así, $f(1.01,1.01) \approx P^1_{f,P}(1.01,1.01) = \frac{-1.01^2-1.01^2+4\cdot 1.01+4\cdot 1.01-6}{2} = 0.0199$.
% \end{frame}
% 
% 
\subsection{Relative extrema}
%---------------------------------------------------------------------slide----
\begin{frame}
\frametitle{Relative extrema of a scalar field}
\begin{definition}[Relative maximum and minimum]
A scalar field $f$ in $\mathbb{R}^n$ has a \emph{relative maximum} at a point $P$ if there is a value $\epsilon>0$ such that 
\[
f(P)\geq f(X)\ \forall X, |\vec{PX}|<\epsilon.
\]

$f$ has a \emph{relative minimum} at $f$ if there is a value $\epsilon>0$ such that
\[
f(P)\leq f(X)\ \forall X, |\vec{PX}|<\epsilon.
\]
Both relative maxima and minima are known as \emph{relative extrema} of $f$.
\end{definition}
\end{frame}


%---------------------------------------------------------------------slide----
\begin{frame}
\frametitle{Critical points}
\begin{theorem}
If a scalar field $f$ in $\mathbb{R}^n$ has a relative maximum or minimum at a point $P$, then $P$ is a \emph{critical or stationary point} of $f$, that is, a point where the gradient vanishes
\[
\nabla f(P) = 0.
\]
\end{theorem}

\structure{\textbf{Proof}} 
Taking the trajectory that passes through $P$ with the direction of the gradient at that point
\[
g(t)=P+t\nabla f(P),
\]
the function $h=(f\circ g)(t)$ does not decrease at $t=0$ since
\[
h'(0)= (f\circ g)'(0) = \nabla f(g(0))\cdot g'(0) = \nabla f(P)\cdot \nabla f(P) = |\nabla f(P)|^2\geq 0,
\]
and it only vanishes if $\nabla f(P)=0$.

Thus, if $\nabla f(P)\neq 0$, $f$ can not have a relative maximum at $P$ since following the trajectory of $g$ from $P$ there are points where $f$ has an image greater than the image at $P$.
In the same way, following the trajectory of $g$ in the opposite direction there are points where $f$ has an image less than the image at $P$, so $f$ can not have relative minimum at $P$. 
\end{frame}


%---------------------------------------------------------------------slide----
\begin{frame}
\frametitle{Critical points}
\framesubtitle{Example}
Given the scalar field $f(x,y)=x^2+y^2$, it is obvious that $f$ only has a relative minimum at $(0,0)$ since
\[
f(0,0)=0 \leq f(x,y)=x^2+y^2,\ \forall x,y\in \mathbb{R}.
\]
Is easy to check that $f$ has a critical point at $(0,0)$, that is $\nabla f(0,0) = 0$.
\begin{center}
\tikzsetnextfilename{derivatives_n_variables/paraboloid_minimum}
%\href{https://www.google.es/search?q=x^2\%2By^2}{
\input{img/derivatives_n_variables/paraboloid_minimum}
%}
\end{center}
\end{frame}


%---------------------------------------------------------------------slide----
\begin{frame}
\frametitle{Saddle points}
Not all the critical points of a scalar field are points where the scalar field has relative extrema. 
If we take, for instance, the scalar field $f(x,y)=x^2-y^2$, its gradient is 
\[
\nabla f(x,y) = (2x,-2y),
\]
that only vanishes at $(0,0)$.
However, this point is not a relative maximum since the points $(x,0)$ in the $x$-axis have images $f(x,0)=x^2\geq
0=f(0,0)$, nor a relative minimum since the points $(0,y)$ in the $y$-axis have images $f(0,y)=-y^2\leq
0=f(0,0)$. 
This type of critical points that are not relative extrema are known as \emph{saddle points}.
\begin{center}
\tikzsetnextfilename{derivatives_n_variables/saddle_point}
%\href{https://www.google.es/search?q=x^2-y^2}{
\input{img/derivatives_n_variables/saddle_point}
%}
\end{center}
\end{frame}


%---------------------------------------------------------------------slide----
\begin{frame}
\frametitle{Analysis of the relative extrema}
From the second degree Taylor's formula of a scalar field $f$ at a point $P$ we have 
\[
f(P+\mathbf{v})-f(P)\approx \nabla f(P)\mathbf{v}+\frac{1}{2}\nabla^2f(P)\mathbf{v}\cdot\mathbf{v}.
\]
Thus, if $P$ is a critical point of $f$, as $\nabla f(P)=0$, we have
\[
f(P+\mathbf{v})-f(P)\approx \frac{1}{2}\nabla^2f(P)\mathbf{v}\cdot\mathbf{v}.
\]
Therefore, the sign of the $f(P+\mathbf{v})-f(P)$ is the sign of the second degree term $\nabla^2f(P)\mathbf{v}\cdot\mathbf{v}$.

There are four possibilities:
\begin{itemize}
\item Definite positive: $\nabla^2f(P)\mathbf{v}\cdot\mathbf{v}>0$ $\forall \mathbf{v}\neq 0$.
\item Definite negative: $\nabla^2f(P)\mathbf{v}\cdot\mathbf{v}<0$ $\forall \mathbf{v}\neq 0$.
\item Indefinite: $\nabla^2f(P)\mathbf{v}\cdot\mathbf{v}>0$ for some $\mathbf{v}\neq 0$ and $\nabla^2f(P)\mathbf{u}\cdot\mathbf{u}<0$ for some $\mathbf{u}\neq 0$.
\item Semidefinite: In any other case. 
\end{itemize}
\end{frame}


%---------------------------------------------------------------------slide----
\begin{frame}
\frametitle{Analysis of the relative extrema}
Thus, depending on de sign of $\nabla^2 f(P)\mathbf{v}\cdot\mathbf{v}$, we have 
\begin{theorem}
Given a critical point $P$ of a scalar field $f$, it holds that
\begin{itemize}
\item If $\nabla^2f(P)$ is definite positive then $f$ has a relative minimum at $P$.
\item If $\nabla^2f(P)$ is definite negative then $f$ has a relative maximum at $P$.
\item If $\nabla^2f(P)$ is indefinite then $f$ has a saddle point at $P$.
\end{itemize}
\end{theorem}
When $\nabla^2f(P)$ is semidefinite we can not draw any conclusion and we need higher order partial derivatives to classify the critical point. 
\end{frame}


%---------------------------------------------------------------------slide----
\begin{frame}
\frametitle{Analysis of the relative extrema of a scalar field in $\mathbb{R}^2$}
In the particular case of a scalar field of two variables, we have
\begin{theorem}
Given a critical point $P=(x_0,y_0)$ of a scalar field $f(x,y)$, it holds that
\begin{itemize}
\item If $Hf(P)>0$ and $\dfrac{\partial^2 f}{\partial x^2}(x_0,y_0)>0$ then $f$ has a relative minimum at $P$.
\item If $Hf(P)>0$ and $\dfrac{\partial^2 f}{\partial x^2}(x_0,y_0)<0$ then $f$ has a relative maximum at $P$.
\item IF $Hf(P)<0$ then $f$ has a saddle point at $P$.
\end{itemize}
\end{theorem}
\end{frame}


%---------------------------------------------------------------------slide----
\begin{frame}
\frametitle{Analysis of the relative extrema of a scalar field in $\mathbb{R}^2$}
\framesubtitle{Example}
Given the scalar field $f(x,y)=\dfrac{x^3}{3}-\dfrac{y^3}{3}-x+y$, its gradient is
\[
\nabla f(x,y)= (x^2-1,-y^2+1),
\]
and it has critical points at $(1,1)$, $(1,-1)$, $(-1,1)$ and $(-1,-1)$.

The hessian matrix is
\[
\nabla^2f(x,y) = \left(
\begin{array}{cc}
2x & 0\\
0 & -2y
\end{array}
\right)
\]
and the hessian is 
\[
Hf(x,y) = -4xy.
\]
Thus, we have 
\begin{itemize}
\item Point $(1,1)$: $Hf(1,1)=-4<0 \Rightarrow$ Saddle point.
\item Point $(1,-1)$: $Hf(1,-1)=4>0$ and $\frac{\partial^2}{\partial x^2}(1,-1)=2>0 \Rightarrow$ Relative min.
\item Point $(-1,1)$: $Hf(-1,1)=4>0$ and $\frac{\partial^2}{\partial x^2}(-1,1)=-2<0 \Rightarrow$ Relative max.
\item Point $(-1,-1)$: $Hf(-1,-1)=-4<0 \Rightarrow$ Saddle point.
\end{itemize}
\end{frame}


%---------------------------------------------------------------------slide----
\begin{frame}
\frametitle{Analysis of the relative extrema of a scalar field in $\mathbb{R}^2$}
\framesubtitle{Example}
Relative extrema and saddle points of the function $f(x,y)=\dfrac{x^3}{3}-\dfrac{y^3}{3}-x+y$.
\begin{center}
\tikzsetnextfilename{derivatives_n_variables/extrema_analysis}
%\href{https://www.google.es/search?q=x^3\%2F3-y^3\%2F3-x\%2By}{
% Author: Alfredo Sánchez Alberca (asalber@ceu.es)
\begin{tikzpicture}
  \begin{axis}[view={120}{20},
  3dfun,
  xmin=-2, xmax=2,
  ymin=-2, ymax=2,
  zmin=-2, zmax=2,
  axis x line=middle,
  axis y line=middle,
  axis z line=middle,
%  axis equal=true,
  every axis x label/.style={at={(xticklabel* cs:1.1)}},
  every axis y label/.style={at={(yticklabel* cs:1.1)}},
  every axis z label/.style={at={(zticklabel* cs:1.1)}},
  x tick label style={anchor=east, inner sep=2pt}, 
  y tick label style={anchor=north, inner sep=5pt},
  z tick label style={anchor=east, inner sep=2pt},  
  mesh/interior colormap name=incolormap,
  colormap name=outcolormap,
  clip=false,
  height=5cm,
  ]
  \addplot3[surf, domain=-2:2, y domain=-2:2, opacity=0.5, faceted color=color1] {x^3/3-y^3/3-x+y};
  \fill[color2] (1,1,0) circle (1.2pt) node[color=DarkTeal, right] {(1,1,0)};
  \fill[color2] (1,-1,-4/3) circle (1.2pt) node[color=DarkTeal, right] {(1,-1,-4/3)};
  \fill[color2] (-1,1,4/3) circle (1.2pt) node[color=DarkTeal, right] {(-1,1,4/3)};
  \fill[color2] (-1,-1,0) circle (1.2pt) node[color=DarkTeal, right] {(-1,-1,0)};
  \end{axis}
\end{tikzpicture}

%}
\end{center}
\end{frame}


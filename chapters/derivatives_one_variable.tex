% Author: Alfredo Sánchez Alberca (asalber@ceu.es)
% !TEX root = ../calculus_manual.tex
\section{Differential calculus with one real variable}
\mode<presentation>{
%---------------------------------------------------------------------slide----
\begin{frame}
\frametitle{Differential calculus with one variable}
\tableofcontents[sectionstyle=show/hide,hideothersubsections]
\end{frame}
}


\subsection{Concept of derivative}
%---------------------------------------------------------------------slide----
\begin{frame}
\frametitle{Increment}
\begin{definition}[Increment of a variable]
An \emph{increment} of a variable $x$ is a change in the value of the variable; is it is denoted $\Delta x$.
The increment of a variable $x$ along an interval $[a,b]$ is given by
\[
\Delta x = b-a.
\]
\end{definition}

\begin{definition}[Increment of a function]
The \emph{increment} of a function $y=f(x)$ along an interval $[a,b]\subseteq Dom(f)$ is given by
\[
\Delta y = f(b)-f(a).
\]
\end{definition}

\textbf{Example} The increment of $x$ along the interval $[2,5]$ is $\Delta x=5-2=3$ and the increment of the function $y=x^2$ along the same interval is $\Delta y=5^2-2^2=21$.
\end{frame}

\begin{frame}
\frametitle{Average rate of change}
The study of a function $y=f(x)$ requires to understand how the function changes, that is, how the dependent variable $y$ changes when we change the independent variable $x$.

\begin{definition}[Average rate of change]
The  \emph{average rate of change} of a function $y=f(x)$ in an interval $[a,a+\Delta x]\subseteq Dom(f)$, is the quotient between the increment of $y$ and the increment of $x$ in that interval, and is denoted by
\[
\mbox{ARC}\;f[a,a+\Delta x]=\frac{\Delta y}{\Delta x}=\frac{f(a+\Delta x)-f(a)}{\Delta x}.
\]
\end{definition}
\end{frame}

%---------------------------------------------------------------------slide----
\begin{frame}
\frametitle{Average rate of change}
\framesubtitle{Example of the area of a square}
Let $y=x^2$ be the function that measures the area of a metallic square of side length $x$.

If at any given time the side of the square is $a$, and we heat the square uniformly increasing the side by dilatation a quantity $\Delta x$, how much will the area of the square increase?
\begin{columns}
\begin{column}{0.3\textwidth}
\begin{align*}
\Delta y &= f(a+\Delta x)-f(a)=(a+\Delta x)^2-a^2=\\
&= a^2+2a\Delta x+\Delta x^2-a^2=2a\Delta x+\Delta x^2.
\end{align*}
\end{column}
\begin{column}{0.3\textwidth}
\begin{center}
\tikzsetnextfilename{derivatives_1_variable/square_area_variation}
\mode<article>{\resizebox{0.3\textwidth}{!}{\begin{tikzpicture}
% \draw (0,0) node[below] {$a$} rectangle (4,4) node[midway] {$a^2$};
% \draw [fill=orange] (0,4) rectangle (4,5) node[midway] {$a\Delta x$};
% \draw [fill=orange] (4,0) rectangle (5,4) node[midway] {$a\Delta x$};
% \draw [fill=orange] (4,4) rectangle (5,5) node[midway] {$\Delta x^2$};
\node (A) [rectangle, draw, minimum width=4cm, minimum height=4cm, label={[anchor=north]south:$a$}] at (0,0) {$a^2$};
\node (B) [anchor=south, rectangle, draw, fill=color4!50, minimum width=4cm, minimum height=1cm] at (A.north) {$a\Delta x$};
\node (C) [anchor=west, rectangle, draw, fill=color4!50, minimum width=1cm, minimum height=4cm, label={[anchor=north]south:$\Delta x$}] at (A.east) {$a\Delta x$};
\node (D) [anchor=south, rectangle, draw, fill=color4!50, minimum width=1cm, minimum height=1cm] at (C.north) {$\Delta x^2$};
\end{tikzpicture}
}}
\mode<presentation>{\resizebox{0.9\textwidth}{!}{\begin{tikzpicture}
% \draw (0,0) node[below] {$a$} rectangle (4,4) node[midway] {$a^2$};
% \draw [fill=orange] (0,4) rectangle (4,5) node[midway] {$a\Delta x$};
% \draw [fill=orange] (4,0) rectangle (5,4) node[midway] {$a\Delta x$};
% \draw [fill=orange] (4,4) rectangle (5,5) node[midway] {$\Delta x^2$};
\node (A) [rectangle, draw, minimum width=4cm, minimum height=4cm, label={[anchor=north]south:$a$}] at (0,0) {$a^2$};
\node (B) [anchor=south, rectangle, draw, fill=color4!50, minimum width=4cm, minimum height=1cm] at (A.north) {$a\Delta x$};
\node (C) [anchor=west, rectangle, draw, fill=color4!50, minimum width=1cm, minimum height=4cm, label={[anchor=north]south:$\Delta x$}] at (A.east) {$a\Delta x$};
\node (D) [anchor=south, rectangle, draw, fill=color4!50, minimum width=1cm, minimum height=1cm] at (C.north) {$\Delta x^2$};
\end{tikzpicture}
}}
\end{center}
\end{column}
\end{columns}
What is the average rate of change in the interval $[a,a+\Delta x]$?
\[
\mbox{ARC}\;f[a,a+\Delta x]=\frac{\Delta y}{\Delta x}=\frac{2a\Delta x+\Delta x^2}{\Delta x}=2a+\Delta x.
\]
\end{frame}


%---------------------------------------------------------------------slide----
\begin{frame}
\frametitle{Geometric interpretation of the average rate of change}
The average rate of change of a function $y=f(x)$ in an interval $[a,a+\Delta x]$ is the slope of the \emph{secant} line to the graph of $f$ through the points $(a,f(a))$ and $(a+\Delta x,f(a+\Delta x))$.
\begin{center}
\tikzsetnextfilename{derivatives_1_variable/secant_line}
% Author: Alfredo Sánchez Alberca (asalber@ceu.es)
\begin{tikzpicture}[trim axis left, trim axis right]
	%\useasboundingbox[draw] (0,0) rectangle (8,5);
	\begin{axis}[
			gen2dfun, 
			xmin=0, xmax=4.5,
			ymin=0, ymax=4,
			axis equal=true,  
			xtick={1,3},
			xticklabels={$a$,$a+\Delta x$},
			ytick={1,2.5},
			yticklabels={$f(a)$,$f(a+\Delta x)$}, 
			%   ticks=none,
			% 	extra y ticks={0.5,0.3},
			% 	extra y tick labels={$f(a)$, $f(a+\Delta x)$},
			% 	extra x ticks={0.5, 3},
			% 	extra x tick labels={$a$,$a+\Delta x$},
			clip=false,
			height=5cm,
		]
		\addplot+[domain=0.2:3.5, smooth, name path=F] {2^(x-2)+0.5} node[anchor=south] {$f(x)$};
		\addplot+[domain=0.2:3.5, smooth, name path=S] {1+1.5/2*(x-1)} node[anchor=west, align=left] {Secant\\ $y=f(a)+\frac{\Delta y}{\Delta x}(x-a)$};
		\fill [name intersections={of=F and S, by={A,B}}]
		(A) circle (1.2pt)
		(B) circle (1.2pt);
		\coordinate (O);
		\draw[gray, dotted] (A) -- (A|-O);
		\draw[gray, dotted] (A) -- (A-|O);
		\draw[gray, dotted] (B) -- (B|-O);
		\draw[gray, dotted] (B) -- (B-|O);
		\draw (A) -- (A-|B) node [midway, below] {$\Delta x$};
		\draw (B) -- (B|-A) node [midway, right] {$\Delta y$};
	\end{axis}
\end{tikzpicture}

\end{center}
\end{frame} 


%---------------------------------------------------------------------slide----
\begin{frame}
\frametitle{Instantaneous rate of change}
Often it is interesting to study the rate of change of a function, not in an interval, but in a point.

Knowing the tendency of change of a function in an instant can be used to predict the value of the function in nearby instants.

\begin{definition}[Instantaneous rate of change and derivative]
The \emph{instantaneous rate of change} of a function $f(x)$ at a point $x=a$, is the limit of the average rate of change of $f$ in the interval $[a,a+\Delta x]$, when $\Delta x$ tends to 0, and is denoted by
\[
\textrm{IRC}\;f (a)=\lim_{\Delta x\rightarrow 0} \textrm{ARC}\; f[a,a+\Delta x]=\lim_{\Delta x\rightarrow 0}\frac{\Delta y}{\Delta x}=\lim_{\Delta x\rightarrow 0}\frac{f(a+\Delta x)-f(a)}{\Delta x}
\]
When this limit exists, the function $f$ is said to be \emph{differentiable} at the point $a$, and its value is called the \emph{derivative} of $f$ at $a$, and it is denoted $f'(a)$ (Lagrange's notation) or $\frac{df}{dx}(a)$ (Leibniz's notation).
\end{definition}
\end{frame}


%---------------------------------------------------------------------slide----
\begin{frame}
\frametitle{Instantaneous rate of change}
\framesubtitle{Example of the area of a square}
Let's take again the function $y=x^2$ that measures the area of a metallic square of side length $x$.

If at any given time the side of the square is $a$, and we heat the square uniformly increasing the side, what is the tendency of change of the area in that moment?
\begin{align*}
\textrm{IRC}\;f(a)&=\lim_{\Delta x\rightarrow 0}\frac{\Delta y}{\Delta x}=\lim_{\Delta x\rightarrow 0}\frac{f(a+\Delta x)-f(a)}{\Delta x} =\\
&=\lim_{\Delta x\rightarrow 0}\frac{2a\Delta x+\Delta x^2}{\Delta x}=\lim_{\Delta x\rightarrow 0} 2a+\Delta x= 2a.
\end{align*}
Thus,
\[
f'(a)=\frac{df}{dx}(a)=2a,
\]
indicating that the area of the square tends to increase the double of the side.
\end{frame}


%---------------------------------------------------------------------slide----
\begin{frame}
\frametitle{Interpretation of the derivative}
The derivative of a function $f'(a)$ shows the growth rate of $f$ at point $a$:
\begin{itemize}
\item $f'(a)>0$ indicates an increasing tendency ($y$ increases as $x$ increases).
\item $f'(a)<0$ indicates a decreasing tendency ($y$ decreases as $x$ increases).
\end{itemize}

\structure{\textbf{Example}} A derivative $f'(a)=3$ indicates that $y$ tends to increase triple of $x$ at point $a$. 
 A derivative $f'(a)=-0.5$ indicates that $y$ tends to decrease half of $x$ at point $a$. 
\end{frame}


%---------------------------------------------------------------------slide----
\begin{frame}
\frametitle{Geometric interpretation of the derivative}
\mode<article>{We have seen that the average rate of change of a function $y=f(x)$ in an interval $[a,a+\Delta x]$ is the slope of the \emph{secant} line, but when $\Delta x$ tends to $0$, the secant line becomes the tangent line.}

The instantaneous rate of change or derivative of a function $y=f(x)$ at $x=a$ is the slope of the \emph{tangent line} to the graph of $f$ at point $(a,f(a))$. 
Thus, the equation of the tangent line to the graph of $f$ at the point $(a,f(a))$ is
\[
y-f(a) = f'(a)(x-a) \Leftrightarrow y = f(a)+f'(a)(x-a)
\]
\begin{center}
\tikzsetnextfilename{derivatives_1_variable/tangent_line}
% Author: Alfredo Sánchez Alberca (asalber@ceu.es)
\begin{tikzpicture}
  \begin{axis}[
    gen2dfun, 
    xmin=0, xmax=4.5,
    ymin=0, ymax=4,
    axis equal=true,  
    xtick={1},
    xticklabels={$a$},
    ytick={1},
    yticklabels={$f(a)$}, 
    height=5cm,
    ]
    \addplot+[domain=0.2:3.5, smooth, name path=F] {2^(x-2)+0.5} node[anchor=south] {$f(x)$};
    \addplot+[domain=0.2:3.5, smooth, name path=T] {1+ln(2)/2*(x-1)} node[anchor=north west, pos=0.8] {Tangent} node [anchor=north, pos=0.8] {$y=f(a)+f'(a)(x-a)$};
    \coordinate (O);
    \coordinate (A) at (1,1);
    \fill (A) circle (1.2pt);
    \draw[gray, dotted] (A) -- (A|-O);
    \draw[gray, dotted] (A) -- (A-|O);
  \end{axis}
\end{tikzpicture}

\end{center}
\end{frame}


% ---------------------------------------------------------------------slide----
\begin{frame}
\frametitle{Kinematic applications: Linear motion}
Assume that the function $y=f(t)$ describes the position of an object moving in the real line at time $t$.
Taking as reference the coordinates origin $O$ and the unitary vector $\mathbf{i}=(1)$, we can represent the position of the moving object $P$ at every moment $t$ with a vector $\vec{OP}=x\mathbf{i}$ where $x=f(t)$.
\begin{center}
\tikzsetnextfilename{derivatives_1_variable/linear_motion}
% Author: Alfredo Sánchez Alberca (asalber@ceu.es)
\begin{tikzpicture}
\draw (0,0) -- (2,0) node[midway, above=0.5] {Time};
\draw (4,0) -- (8,0) node[midway, above=0.5] {Position} node[right] {$\mathbb{R}$};
\draw (1,0.1) -- (1,-0.1) node[anchor=north] (A) {$t$};
\draw (5,0.1) -- (5,-0.1) node[anchor=north] {$O$};
\draw (6,0.1) -- (6,-0.1) node[anchor=north] {$1$};
\draw [->, color1] (5,0) -- (7,0);
\draw [->, color2] (5,0) -- (6,0) node[midway, above] {\color{color2}$\mathbf{i}$};
\fill (7,0) circle (1.2pt) node[above] {$P$} node[anchor=north, xshift=2.5ex] (P) {$x=f(t)$};
\path [->, dashed] (A) edge[bend right] node[above] {$f$} (P);
\end{tikzpicture}

\end{center}

\structure{\textbf{Remark}} It also makes sense when $f$ measures other magnitudes as the temperature of a body, the concentration of a gas, or the quantity of substance in a chemical reaction at every moment $t$.
\end{frame}


% ---------------------------------------------------------------------slide----
\begin{frame}
\frametitle{Kinematic interpretation of the average rate of change}
In this context, if we take the instants $t=a$ and $t=a+\Delta t$, both in $\mbox{Dom}(f)$, the vector
\[
\mathbf{v}_m=\frac{f(a+\Delta t)-f(a)}{\Delta t}
\]
is known as the \emph{average velocity} of the trajectory $f$ in the interval $[a, a+\Delta t]$.

\structure{\textbf{Example}}
A vehicle makes a trip from Madrid to Barcelona.
Let $f(t)$ be the function that determine the position of the vehicle at every moment $t$.
If the vehicle departs from Madrid (km 0) at 8:00 and arrives at Barcelona (km 600) at 14:00, then the average velocity
of the vehicle in the path is
\[
\mathbf{v}_m=\frac{f(14)-f(8)}{14-8}=\frac{600-0}{6} = 100 km/h.
\]
\end{frame}


% ---------------------------------------------------------------------slide----
\begin{frame}
\frametitle{Kinematic interpretation of the derivative}
In the same context of the linear motion, the derivative of the function $f(t)$ at the moment $t_0$ is the vector
\[
\mathbf{v}=f'(a)=\lim_{\Delta t\rightarrow 0}\frac{f(a+\Delta t)-f(a)}{\Delta t},
\]
that is known, as long as the limit exists, as the \emph{instantaneous velocity} or simply \emph{velocity} of the trajectory $f$ at moment $a$.

That is, the derivative of the object position with respect to time is a vector field that is called \emph{velocity along the trajectory $f$}.

\structure{\textbf{Example}}
Following with the previous example, what indicates the speedometer at any instant is the modulus of the instantaneous velocity vector at that moment.
\end{frame}


\subsection{Algebra of derivatives}
%---------------------------------------------------------------------slide----
\begin{frame}
\frametitle{Properties of the derivative}
If $y=c$, is a constant function, then $y'=0$ at any point.

If $y=x$, is the identity function, then  $y'=1$ at any point.

If $u=f(x)$ and $v=g(x)$ are two differentiable functions, then 
\begin{itemize}
\item $(u+v)'=u'+v'$
\item $(u-v)'=u'-v'$
\item $(u\cdot v)'=u'\cdot v+ u\cdot v'$
\item $\left(\dfrac{u}{v}\right)'=\dfrac{u'\cdot v-u\cdot v'}{v^2}$
\end{itemize}
\end{frame}


%---------------------------------------------------------------------slide----
\begin{frame}
\frametitle{Derivative of a composite function}
\framesubtitle{The chain rule}
\begin{theorem}[Chain rule] If the function $y=f\circ g$ is the composition of two functions $y=f(z)$ and $z=g(x)$, then
\[ 
(f\circ g)'(x)=f'(g(x))g'(x).
\]
\end{theorem}

It is easy to prove this fact using the Leibniz notation
\[
\frac{dy}{dx}=\frac{dy}{dz}\frac{dz}{dx}=f'(z)g'(x)=f'(g(x))g'(x).
\]

\structure{\textbf{Example}} If $f(z)=\sin z$ and $g(x)=x^2$, then $f\circ g(x)=\sin(x^2)$. Applying the chain rule the derivative of the composite function is
\[
(f\circ g)'(x)=f'(g(x))g'(x) = \cos(g(x)) 2x = \cos(x^2)2x.
\]
On the other hand, $g\circ f(z)= (\sin z)^2$, and applying the chain rule again, its derivative is
\[
(g\circ f)'(z)=g'(f(z))f'(z) = 2f(z)\cos z = 2\sin z\cos z.
\]
\end{frame}


%---------------------------------------------------------------------slide----
\begin{frame}
\frametitle{Derivative of the inverse of a function}
\begin{theorem}[Derivative of the inverse function]
Given a function $y=f(x)$ with inverse $x=f^{-1}(y)$, then 
\[
\left(f^{-1}\right)'(y)=\frac{1}{f'(x)}=\frac{1}{f'(f^{-1}(y))},
\]
provided that $f$ is differentiable at $f^{-1}(y)$ and $f'(f^{-1}(y))\neq 0$.
\end{theorem}

Again, it is easy to prove this equality using the Leibniz notation
\[
\frac{dx}{dy}=\frac{1}{dy/dx}=\frac{1}{f'(x)}=\frac{1}{f'(f^{-1}(y))}
\]
\end{frame}


%---------------------------------------------------------------------slide----
\begin{frame}
\frametitle{Derivative of the inverse of a function}
\framesubtitle{Example}
The inverse of the exponential function $y=f(x)=e^x$ is the natural logarithm $x=f^{-1}(y)=\ln y$, so we can compute the derivative of the natural logarithm using the previous theorem and we get
\[
\left(f^{-1}\right)'(y)=\frac{1}{f'(x)}=\frac{1}{e^x}=\frac{1}{e^{\ln y}}=\frac{1}{y}.
\]

\structure{\textbf{Example}} Sometimes it is easier to apply the chain rule to compute the derivative of the inverse of a function. 
In this example, as $\ln x$ is the inverse of $e^x$, we know that $e^{\ln x}=x$, so differentiating both sides and applying the chain rule to the left side we get
\[
(e^{\ln x})'=x' \Leftrightarrow e^{\ln x}(\ln(x))' = 1 \Leftrightarrow (\ln(x))'=\frac{1}{e^{\ln x}}=\frac{1}{x}.
\]
\end{frame}



\subsection{Function approximation}

%---------------------------------------------------------------------slide----
\begin{frame}
\frametitle{Approximating a function with the derivative}
The tangent line to the graph of a function $f(x)$ at $x=a$ can be used to approximate $f$ in a neighbourhood of $a$.
\begin{center}
\tikzsetnextfilename{derivatives_1_variable/tangent_line_approximation}
% Author: Alfredo Sánchez Alberca (asalber@ceu.es)
\begin{tikzpicture}[trim axis left, trim axis right]
  \begin{axis}[
    gen2dfun, 
    xmin=0, xmax=4.5,
    ymin=0, ymax=3.5,
    xtick={1,3},
    xticklabels={$a$, $a+\Delta x$},
    ytick={1,2.5},
    yticklabels={$f(a)$,$f(a+\Delta x)$},
    clip=false,
    height=4cm,
    ]
    \addplot+[domain=0.2:3.5, smooth] {2^(x-2)+0.5} node[anchor=south] {$f(x)$};
    \coordinate (O);
    \coordinate (A) at (1,1);
    \coordinate (B) at (3,2.5);
    \coordinate (C) at (3,1.693147);
    \fill (A) circle (1.2pt);
    \draw[gray, dotted] (A) -- (A|-O);
    \draw[gray, dotted] (A) -- (A-|O);
    \draw[gray, dotted] (B) -- (B|-O);
    \draw[gray, dotted] (B) -- (B-|O);
    \draw<2->[gray, dashed] (A) -- (A-|C);
    \draw<2->[decorate, decoration={brace, amplitude=5pt, mirror}] (A) -- (C|-A) node [midway, below, yshift=-4pt] {$\Delta x$};    
    \draw<3->[gray, dashed] (C) -- (C|-A);
    \draw<3->[decorate, decoration={brace, amplitude=5pt, mirror}] (B) -- (B|-A) node [midway, left, xshift=-4pt] {$\Delta y$};
    \addplot+[domain=0.2:3.5, smooth, visible on=<4->] {1+ln(2)/2*(x-1)} node [anchor=west] {$y=f(a)+f'(a)(x-a)$ Tangent};
    \fill<5-> (C) circle (1.2pt);
    \draw<5->[gray, dotted] (C) -- (C-|O);
    \draw<5->[gray, very thin] (-0.05,1.693147) -- (0.05,1.693147);
    \node<5->[anchor=east] at (-0.05,1.693147) {$f(a)+f'(a)\Delta x$};
    \draw<6->[decorate, decoration={brace, amplitude=4pt}] (C) -- (C|-A) node [midway, right, xshift=4pt] {$f'(a)\Delta x$};
    \draw<7->[decorate, decoration={brace, amplitude=4pt}] (B) -- (C) node [midway, right, xshift=4pt, color2] {Error};
  \end{axis}
\end{tikzpicture}

\end{center}
\only<8->{
Thus, the increment of a function $f(x)$ in an interval $[a,a+\Delta x]$ can be approximated multiplying the derivative of $f$ at $a$ by the increment of $x$
\[
\Delta y \approx f'(a)\Delta x
\]
}
\end{frame}


%---------------------------------------------------------------------slide----
\begin{frame}
\frametitle{Approximating a function with the derivative}
\framesubtitle{Example of the area of a square}
In the previous example of the function $y=x^2$ that measures the area of a metallic square of side $x$, if the side of the square is $a$ and we increment it by a quantity $\Delta x$, then the increment on the area will be approximately
\[
\Delta y \approx f'(a)\Delta x = 2a\Delta x.
\]
In the figure below we can see that the error of this approximation is $\Delta x^2$, which is smaller than $\Delta x$ when $\Delta x$ tends to 0. 
\begin{center}
\tikzsetnextfilename{derivatives_1_variable/square_area_variation_approximation}
\mode<article>{\resizebox{0.3\textwidth}{!}{\input{img/derivatives_1_variable/square_area_variation_approximation}}}
\mode<presentation>{\resizebox{0.35\textwidth}{!}{\input{img/derivatives_1_variable/square_area_variation_approximation}}}
\end{center}
\end{frame}


%---------------------------------------------------------------------slide----
\begin{frame}
\frametitle{Approximating a function by a polynomial}
Another useful application of the derivative is the approximation of functions by polynomials.

Polynomials are functions easy to calculate (sums and products) with very good properties:
\begin{itemize}
\item Defined in all the real numbers.
\item Continuous.
\item Differentiable of all orders with continuous derivatives.
\end{itemize}

\begin{block}{Goal}
Approximate a function $f(x)$ by a polynomial $p(x)$ near a point $x=a$.
\end{block}
\end{frame}


%---------------------------------------------------------------------slide----
\begin{frame}
\frametitle{Approximating a function by a polynomial of order 0}
A polynomial of degree 0 has equation
\[
p(x) = c_0,
\]
where $c_0$ is a constant.

As the polynomial should coincide with the function $f$ at $a$, it must satisfy
\[p(a) = c_0 = f(a).\]

Therefore, the polynomial of degree 0 that best approximates $f$ near $a$ is
\[p(x) = f(a).\]
\end{frame}


%---------------------------------------------------------------------slide----
\begin{frame}
\frametitle{Approximating a function by a polynomial of order 0}
\begin{center}
\tikzsetnextfilename{derivatives_1_variable/approximation_polynomial_0}
% Author: Alfredo Sánchez Alberca (asalber@ceu.es)
\begin{tikzpicture}[trim axis left, trim axis right]
  \begin{axis}[
    gen2dfun, 
    xmin=0, xmax=4,
    ymin=0, ymax=4,
    xtick={2.5},
    xticklabel={$a$},
    ytick={1.5},
    yticklabels={$f(a)$}, 
    clip=false, 
    height=5cm,
  	]
    \addplot+[domain=0.5:3.6, smooth, name path=F] {2.7183^(x-2.5)+0.5} node[anchor=south west] {$f(x)$};
    \coordinate (O);
    \coordinate (A) at (2.5,1.5);
    \fill (A) circle (1.2pt);
    \draw[gray, dotted] (A) -- (A|-O);
    \draw[gray, dotted] (A) -- (A-|O);
    \addplot+[domain=0.5:3.6, smooth, name path=P, visible on=<2->] {1.5} node[anchor=west] {$p^0=f(a)$};
    \coordinate (B) at (3.5,3.2183);
    \draw[gray, very thin, visible on=<3->] (3.5,0.05) -- (3.5,-0.05);
    \node[yshift=-1.5ex, visible on=<3->] at (3.5,0) {$x$};
    \draw[gray, very thin, visible on=<3->] (-0.05,3.2183) -- (0.05,3.2183);
    \node[anchor=east, xshift=-0.25em, visible on=<3->] at (0,3.2183) {$f(x)$};
    \draw[gray, dotted, visible on=<3->] (B) -- (B|-O);
    \draw[gray, dotted, visible on=<3->] (B) -- (B-|O);
    \draw[dashed, visible on=<4->] (B) -- (B|-A);
    \draw[decorate, decoration={brace, amplitude=4pt}, visible on=<4->] (B) -- (B|-A) node [pos=0.3, right, xshift=5pt] {Approximation error} node[midway, right, xshift=4pt] {$e^0(x)=f(x)-p^0(x)$};
  \end{axis};
\end{tikzpicture}

\end{center}
\end{frame}


%---------------------------------------------------------------------slide----
\begin{frame}
\frametitle{Approximating a function by a polynomial of order 1}
A polynomial of degree 1 has equation
\[
p(x) = c_0+c_1x,
\]
but it can also be written as 
\[
p(x) = c_0+c_1(x-a).
\]

Among all the polynomials of degree 1, the one that best approximates $f$ near $a$ is that which meets the following conditions

\begin{enumerate}
\item $p$ and $f$ coincide at $a$: $p(a) = f(a)$,
\item $p$ and $f$ have the same rate of change at $a$: $p'(a) = f'(a)$.
\end{enumerate}

The last condition guarantees that $p$ and $f$ have approximately the same tendency, but it requires the function $f$ to be differentiable at $a$.
\end{frame}


%---------------------------------------------------------------------slide----
\begin{frame}
\frametitle{The tangent line: Best approximating polynomial of order 1}
Imposing the previous conditions we have
\begin{enumerate}
\item $p(x)=c_0+c_1(x-a) \Rightarrow p(a)=c_0+c_1(a-a)=c_0=f(a)$,
\item $p'(x)=c_1 \Rightarrow p'(a)=c_1=f'(a)$.
\end{enumerate}

Therefore, the polynomial of degree 1 that best approximates $f$ near $a$ is
\[
p(x) = f(a)+f '(a)(x-a),
\]
which turns out to be the tangent line to $f$ at $(a,f(a))$.
\end{frame}


%---------------------------------------------------------------------slide----
\begin{frame}
\frametitle{Approximating a function by a polynomial of order 1}
\begin{center}
\tikzsetnextfilename{derivatives_1_variable/approximation_polynomial_1}
% Author: Alfredo Sánchez Alberca (asalber@ceu.es)
\begin{tikzpicture}[trim axis left, trim axis right]
  \begin{axis}[
    gen2dfun, 
    xmin=0, xmax=4,
    ymin=0, ymax=4,
    xtick={2.5},
    xticklabel={$a$},
    ytick={1.5},
    yticklabels={$f(a)$}, 
    clip=false, 
    height=5cm,
  	]
    \addplot+[domain=0.5:3.6, smooth, name path=F] {2.7183^(x-2.5)+0.5} node[anchor=south west] {$f(x)$};
    \coordinate (O);
    \coordinate (A) at (2.5,1.5);
    \fill (A) circle (1.2pt);
    \draw[gray, dotted] (A) -- (A|-O);
    \draw[gray, dotted] (A) -- (A-|O);
    \addplot+[domain=0.5:3.6, smooth, name path=P0, visible on=<2->] {1.5} node[anchor=west] {$p^0=f(a)$};
    \addplot+[domain=1:3.6, smooth, name path=P1, visible on=<3->] {x-1} node[anchor=west] {$p^1=f(a)+f'(a)(x-a)$};
    \coordinate (B) at (3.5,3.2183);
    \coordinate (C) at (3.5,2.5);
    \draw<4->[gray, very thin] (3.5,0.05) -- (3.5,-0.05);
    \node<4->[anchor=north] at (3.5,-0.05) {$x$};
    \draw<4->[gray, very thin] (-0.05,3.2183) -- (0.05,3.2183);
    \node<4->[anchor=east] at (-0.05,3.2183) {$f(x)$};
    \draw<4->[gray, very thin] (-0.05,2.5) -- (0.05,2.5);
    \node<4->[anchor=east] at (-0.05,2.5) {$p^1(x)$};
    \draw<4->[gray, dotted] (B) -- (B-|O);
    \draw<4->[gray, dotted] (B) -- (B|-O);
    \draw<4->[gray, dotted] (C) -- (C-|O);
    \draw[|-|, visible on=<5->] (B) -- (C) node[anchor=west, pos=0.1] {Approximation error} node[anchor=west, midway] {$e^1(x)=f(x)-p^1(x)$};
  \end{axis};
\end{tikzpicture}

\end{center}
\end{frame}


%---------------------------------------------------------------------slide----
\begin{frame}
\frametitle{Approximating a function by a polynomial of order 2}
A polynomial of degree 2 is a parabola with equation
\[
p(x) = c_0+c_1x+c_2x^2,
\]
but it can also be written as
\[
p(x) = c_0+c_1(x-a)+c_2(x-a)^2.
\]

Among all the polynomials of degree 2, the one that best approximate $f(x)$ near $a$ is that which meets the following conditions
\begin{enumerate}
\item $p$ and $f$ coincide at $a$: $p(a) = f(a)$,
\item $p$ and $f$ have the same rate of change at $a$: $p'(a) = f'(a)$.
\item $p$ and $f$ have the same concavity at $a$: $p''(a)=f''(a)$.
\end{enumerate}
The last condition requires the function $f$ to be differentiable twice at $a$.
\end{frame}


%---------------------------------------------------------------------slide----
\begin{frame}
\frametitle{Best approximating polynomial of order 2}
Imposing the previous conditions we have
\begin{enumerate}
\item $p(x)=c_0+c_1(x-a) \Rightarrow p(a)=c_0+c_1(a-a)=c_0=f(a)$,
\item $p'(x)=c_1 \Rightarrow p'(a)=c_1=f'(a)$.
\item $p''(x)=2c_2 \Rightarrow p''(a)=2c_2=f''(a) \Rightarrow c_2=\frac{f''(a)}{2}$.
\end{enumerate}

Therefore, the polynomial of degree 2 that best approximates $f$ near $a$ is
\[
p(x) = f(a)+f'(a)(x-a)+\frac{f''(a)}{2}(x-a)^2.
\]
\end{frame}


%---------------------------------------------------------------------slide----
\begin{frame}
\frametitle{Approximating a function by a polynomial of order 2}
\begin{center}
\tikzsetnextfilename{derivatives_1_variable/approximation_polynomial_2}
% Author: Alfredo Sánchez Alberca (asalber@ceu.es)
\begin{tikzpicture}[trim axis left, trim axis right]
  \begin{axis}[
    gen2dfun, 
    xmin=0, xmax=4,
    ymin=0, ymax=4,
    xtick={2.5},
    xticklabel={$a$},
    ytick={1.5},
    yticklabels={$f(a)$}, 
    clip=false, 
    height=5cm,
  	]
    \addplot+[domain=0.5:3.6, smooth, name path=F] {2.7183^(x-2.5)+0.5} node[anchor=south west] {$f(x)$};
    \coordinate (O);
    \coordinate (A) at (2.5,1.5);
    \fill (A) circle (1.2pt);
    \draw[gray, dotted] (A) -- (A|-O);
    \draw[gray, dotted] (A) -- (A-|O);
    \addplot+[domain=0.5:3.6, smooth, name path=P0, visible on=<2->] {1.5} node[anchor=west] {$p^0=f(a)$};
    \addplot+[domain=1:3.6, smooth, name path=P1, visible on=<3->] {x-1} node[anchor=west] {$p^1=f(a)+f'(a)(x-a)$};
    \addplot+[domain=0.5:3.6, smooth, name path=P1, visible on=<4->] {-1+x+(x-2.5)^2/2} node[anchor=west] {$p^2=f(a)+f'(a)(x-a)+\frac{f''(a)}{2}(x-a)^2$};
    \coordinate (B) at (3.5,3.2183);
    \coordinate (C) at (3.5,3);
    \draw<4->[gray, very thin] (3.5,0.05) -- (3.5,-0.05);
    \node<4-> at (3.5,-0.25) {$x$};
    \draw<4->[gray, very thin] (-0.06,3.2183) -- (0.06,3.2183);
    \node<4-> at (-0.25,3.2183) {$f(x)$};
    \draw<4->[gray, dotted] (B) -- (B|-O);
    \draw<4->[gray, dotted] (B) -- (B-|O);
    \draw[dashed, visible on=<5->] (B) -- (B|-A);
    \draw[decorate, decoration={brace, amplitude=4pt}, visible on=<5->] (B) -- (B|-A) node [pos=0.3, right, xshift=5pt] {Approximation error} node[midway, right, xshift=4pt] {$e^2(x)=f(x)-p^2(x)$};
  \end{axis};
\end{tikzpicture}

\end{center}
\end{frame}


%---------------------------------------------------------------------slide----
\begin{frame}
\frametitle{Approximating a function by a polynomial of order $n$}
A polynomial of degree $n$ has equation
\[
p(x) = c_0+c_1x+c_2x^2+\cdots +c_nx^n,
\]
but it can also be written as
\[
p(x) = c_0+c_1(x-a)+c_2(x-a)^2+\cdots +c_n(x-a)^n.
\]

Among all the polynomials of degree 2, the one that best approximate $f(x)$ near $a$ is that which meets the following $n+1$ conditions
\begin{enumerate}
\item $p(a) = f(a)$,
\item $p'(a) = f'(a)$,
\item $p''(a)=f''(a)$,
\item[] $\cdots$
\item[n+1.] $p^{(n)}(a)=f^{(n)}(a)$.
\end{enumerate}

\alert{Observe that these conditions require the function $f$ to be differentiable $n$ times at $a$.}
\end{frame}


%---------------------------------------------------------------------slide----
\begin{frame}
\frametitle{Coefficients calculation for the best approximating polynomial of order $n$}
The successive derivatives of $p$ are 
\begin{align*}
p(x) &= c_0+c_1(x-a)+c_2(x-a)^2+\cdots +c_n(x-a)^n,\\
p'(x)& = c_1+2c_2(x-a)+\cdots +nc_n(x-a)^{n-1},\\
p''(x)& = 2c_2+\cdots +n(n-1)c_n(x-a)^{n-2},\\
\vdots\ \
\\
p^{(n)}(x)&= n(n-1)(n-2)\cdots 1 c_n=n!c_n.
\end{align*}

Imposing the previous conditions we have
\begin{enumerate}
\item $p(a) = c_0+c_1(a-a)+c_2(a-a)^2+\cdots +c_n(a-a)^n=c_0=f(a)$,
\item $p'(a) = c_1+2c_2(a-a)+\cdots +nc_n(a-a)^{n-1}=c_1=f'(a)$,
\item $p''(a) = 2c_2+\cdots +n(n-1)c_n(a-a)^{n-2}=2c_2=f''(a)\Rightarrow c_2=f''(a)/2$,
\item[] $\cdots$
\item[n+1.] $p^{(n)}(a)=n!c_n=f^{(n)}(a)=c_n=\frac{f^{(n)}(a)}{n!}$.
\end{enumerate}
\end{frame}


%---------------------------------------------------------------------slide----
\begin{frame}
\frametitle{Taylor polynomial of order $n$}
\begin{definition}[Taylor polynomial]
Given a function $f(x)$ differentiable $n$ times at $a$, the \emph{Taylor polynomial} of order $n$ of $f$ at $a$ is the polynomial with equation
\begin{align*}
p_{f,a}^n(x) &= f(a) + f'(a)(x-a) + \frac{f''(a)}{2}(x-a)^2 + \cdots + \frac{f^{(n)}(a)}{n!}(x-a)^n = \\ 
&= \sum_{i=0}^{n}\frac{f^{(i)}(a)}{i!}(x-a)^i.
\end{align*}
\end{definition}

The Taylor polynomial of order $n$ of $f$ at $a$ is the $n$th degree polynomial that best approximates $f$ near $a$, as is the only one that meets the previous conditions.
\end{frame} 


%---------------------------------------------------------------------slide----
\begin{frame}
\frametitle{Taylor polynomial calculation}
\framesubtitle{Example}
Let us approximate the function $f(x)=\log x$ near the value $1$ by a polynomial of order $3$.

The equation of the Taylor polynomial of order $3$ of $f$ at $a=1$ is
\[
p_{f,1}^3(x)=f(1)+f'(1)(x-1)+\frac{f''(1)}{2}(x-1)^2+\frac{f'''(1)}{3!}(x-1)^3.
\]
The derivatives of $f$ at $1$ up to order $3$ are
\[
\begin{array}{lll}
f(x)=\log x & \quad & f(1)=\log 1 =0,\\
f'(x)=1/x & & f'(1)=1/1=1,\\
f''(x)=-1/x^2 & & f''(1)=-1/1^2=-1,\\
f'''(x)=2/x^3 & & f'''(1)=2/1^3=2.
\end{array}
\]
And substituting into the polynomial equation we get
\[
p_{f,1}^3(x)=0+1(x-1)+\frac{-1}{2}(x-1)^2+\frac{2}{3!}(x-1)^3= \frac{2}{3}x^3-\frac{3}{2}x^2+3x-\frac{11}{6}.
\]
\end{frame}


%---------------------------------------------------------------------slide----
\begin{frame}
\frametitle{Taylor polynomials of the logarithmic function}
\begin{center}
\tikzsetnextfilename{derivatives_1_variable/taylor_polynomials_logarithm}
% Author: Alfredo Sánchez Alberca (asalber@ceu.es)
\begin{tikzpicture}
  \begin{axis}[
    2dfun, 
    xmin=0, xmax=4,
    ymin=-2, ymax=2,
    clip=false, 
    height=5cm,
  	]
    \addplot+[domain=0.2:2.8, smooth, samples=200] {ln(x)} node[anchor=west] {$f(x)=\log(x)$};
    \addplot+[domain=0.2:2.8, smooth, visible on=<2->] {x-1} node[anchor=west] {$p^1_{f,1}(x)=-1+x$};
    \addplot+[domain=0.2:2.8, smooth, visible on=<3->] {-1+x-(x-1)^2/2} node[anchor=west] {$p_{f,1}^2(x)=-1+x-\frac{1}{2}(x-1)^2$};
    \addplot+[domain=0.2:2.8, smooth, visible on=<4->] {-1+x-(x-1)^2/2+2/6*(x-1)^3} node[anchor=south west] {$p_{f,1}^3(x)=-1+x-\frac{1}{2}(x-1)^2+\frac{1}{3}(x-1)^3$};
    \coordinate (A) at (1,0);
    \fill (A) circle (1.2pt);
  \end{axis};
\end{tikzpicture}

\end{center}
\end{frame}


%---------------------------------------------------------------------slide----
\begin{frame}
\frametitle{Maclaurin polynomial of order $n$}
The Taylor polynomial equation has a simpler form when the polynomial is calculated at $0$.
This special case of Taylor polynomial at $0$ is known as the \emph{Maclaurin polynomial}.
\begin{definition}[Maclaurin polynomial]
Given a function $f(x)$ differentiable $n$ times at $0$, the \emph{Maclaurin polynomial} of order $n$ of $f$ is the polynomial with equation
\begin{align*}
p_{f,0}^n(x)&=f(0)+f'(0)x+\frac{f''(0)}{2}x^2+\cdots +\frac{f^{(n)}(0)}{n!}x^n = \\ &=\sum_{i=0}^{n}\frac{f^{(i)}(0)}{i!}x^i.
\end{align*}
\end{definition}
\end{frame}


%---------------------------------------------------------------------slide----
\begin{frame}
\frametitle{Maclaurin polynomial calculation}
\framesubtitle{Example}
Let us approximate the function $f(x)=\sin x$ near the value $0$ by a polynomial of order $3$.

The Maclaurin polynomial equation of order $3$ of $f$ is
\[
p_{f,0}^3(x)=f(0)+f'(0)x+\frac{f''(0)}{2}x^2+\frac{f'''(0)}{3!}x^3.
\]
The derivatives of $f$ at $0$ up to order $3$ are
\[
\begin{array}{lll}
f(x)=\sin x & \quad & f(0)=\sin 0 =0,\\
f'(x)=\cos x & & f'(0)=\cos 0=1,\\
f''(x)=-\sin x & & f''(0)=-\sin 0=0,\\
f'''(x)=-\cos x & & f'''(0)=-\cos 0=-1.
\end{array}
\]
And substituting into the polynomial equation we get
\[
p_{f,0}^3(x)=0+1\cdot x+\frac{0}{2}x^2+\frac{-1}{3!}x^3= x-\frac{x^3}{6}.
\]
\end{frame}


%---------------------------------------------------------------------slide----
\begin{frame}
\frametitle{Maclaurin polynomial of the sine function}
\begin{center}
\tikzsetnextfilename{derivatives_1_variable/maclaurin_polynomials_sine}
\input{img/derivatives_1_variable/maclaurin_polynomials_sine}
\end{center}
\end{frame}


%---------------------------------------------------------------------slide----
\begin{frame}
\frametitle{Maclaurin polynomials of elementary functions}
\[
\renewcommand{\arraystretch}{2.5}
\begin{array}{cc}
\toprule
f(x) & p_{f,0}^n(x) \\
\midrule
\sin x & \displaystyle x - \frac{x^3}{3!} + \frac{x^5}{5!} - \cdots + (-1)^k\frac{x^{2k-1}}{(2k-1)!} \mbox{ if $n=2k$ or $n=2k-1$}\\
\cos x &  \displaystyle 1 - \frac{x^2}{2!} + \frac{x^4}{4!} - \cdots + (-1)^k\frac{x^{2k}}{(2k)!} \mbox{ if $n=2k$ or $n=2k+1$}\\
\arctan x &  \displaystyle x - \frac{x^3}{3} + \frac{x^5}{5} - \cdots + (-1)^k\frac{x^{2k-1}}{(2k-1)} \mbox{ if $n=2k$ or $n=2k-1$}\\
e^x & \displaystyle 1 + x + \frac{x^2}{2!} + \frac{x^3}{3!} + \cdots + \frac{x^n}{n!}\\
\log(1+x) & \displaystyle x - \frac{x^2}{2} + \frac{x^3}{3} - \cdots + (-1)^{n-1}\frac{x^n}{n}\\
\bottomrule
\end{array}
\]
\end{frame}


%---------------------------------------------------------------------slide----
\begin{frame}
\frametitle{Taylor remainder and Taylor formula}
Taylor polynomials allow to approximate a function in a neighborhood of a value $a$, but most of the times there is an error in the approximation.
\begin{definition}[Taylor remainder]
Given a function  $f(x)$ and its Taylor polynomial of order $n$ at $a$, $p_{f,a}^n(x)$, the \emph{Taylor remainder} of order $n$ of $f$ at $a$ is the difference between the function and the polynomial,
\[
r_{f,a}^n(x)=f(x)-p_{f,a}^n(x).
\]
\end{definition}

The Taylor remainder measures the error in the approximation of $f(x)$ by the Taylor polynomial and allow us to express the function as the Taylor polynomial plus the Taylor remainder
\[
f(x)=p_{f,a}^n(x) + r_{f,a}^n(x).
\]
This expression is known as the \emph{Taylor formula} of order $n$ or $f$ at $a$. 

It can be proved that
\[
\lim_{h\rightarrow 0}\frac{r_{f,a}^n(a+h)}{h^n}=0,
\]
which means that the remainder $r_{f,a}^n(a+h)$ is much smaller than $h^n$.
\end{frame}


\subsection{Analysis of functions}
%---------------------------------------------------------------------slide----
\begin{frame}
\frametitle{Analysis of functions: increase and decrease}
The main application of derivatives is to determine the variation (increase or decrease) of functions. 
For that we use the sign of the first derivative.  
\begin{theorem}
Let $f(x)$ be a function with first derivative in an interval $I\subseteq \mathbb{R}$.
\begin{itemize}
\item If $\forall x\in I\ f'(x)> 0$ then $f$ is increasing on $I$.
\item If $\forall x\in I\ f'(x)< 0$ then $f$ is decreasing on $I$.
\end{itemize}
\end{theorem}
If $f'(a)=0$ then $a$ is known as a \emph{critical point}.
At this point the function can be increasing, decreasing or neither increasing nor decreasing. 

\structure{\textbf{Example}}
The function $f(x)=x^2$ has derivative $f'(x)=2x$; it is decreasing on $\mathbb{R}^-$ as $f'(x)< 0$ $\forall x\in \mathbb{R}^-$  and increasing on $\mathbb{R}^+$ as $f'(x)> 0$ $\forall x\in \mathbb{R}^+$.

It has a critical point at $x=0$, as $f'(0)=0$; at this point the function is neither increasing nor decreasing.

\structure{\textbf{Remark}} {A function can be increasing or decreasing on an interval and not have first derivative.}
\end{frame}


%---------------------------------------------------------------------slide----
\begin{frame}
\frametitle{Analysis of functions: increase and decrease}
\framesubtitle{Example}
Let us analyze the increase and decrease of the function $f(x)=x^4-2x^2+1$. 
Its first derivative is $f'(x)=4x^3-4x$.
\begin{center}
\tikzsetnextfilename{derivatives_1_variable/increase_analysis}
\scalebox{0.9}{% Author: Alfredo Sánchez Alberca (asalber@ceu.es)
\begin{tikzpicture}
  \begin{axis}[
    2dfun, 
    xmin=-2, xmax=2,
    ymin=-2, ymax=2,
    axis equal=true,
    clip=false, 
    width=5cm,
    ]
    \addplot+[domain=-1.5:1.5, smooth, name path=F] {x^4-2*x^2+1} node[anchor=south west] {$f(x)=x^4-2x^2+1$};
    \addplot+[domain=-1.2:1.2, smooth, name path=D] {4*x^3-4*x)} node[anchor=south west] {$f'(x)=4x^3-4x$};
    \node[anchor=east, visible on=<2->] at (-2,-2.7) {Increase $f(x)$};
    \node[anchor=east, visible on=<2->] at (-2,-3.5) {Sign $f'(x)$};
    \draw<3->[dashed, gray] (-1,2) -- (-1,-3.5);
    \draw<3->[dashed, gray] (0,2) -- (0,-3.5);
    \draw<3->[dashed, gray] (1,2) -- (1,-3.5);
    \node<3->[color2] at (-1,-3.5) {$0$};
    \node<3->[color2] at (0,-3.5) {$0$};
    \node<3->[color2] at (1,-3.5) {$0$};
    \node<4->[color2] at (-1.5,-3.5) {$-$};
    \draw<4-> [->, color1] (-1.5, -2.5) -- (-1.5,-2.8);
    \node<5->[color2] at (-0.5,-3.5) {$+$};
    \draw<5-> [->, color1] (-0.5, -2.8) -- (-0.5,-2.5);
    \node<6->[color2] at (0.5,-3.5) {$-$};
    \draw<6-> [->, color1] (0.5, -2.5) -- (0.5,-2.8);   
    \node<7->[color2] at (1.5,-3.5) {$+$};
    \draw<7-> [->, color1] (1.5, -2.8) -- (1.5,-2.5);
  \end{axis}
  \begin{axis}[visible on=<2->,
    2dfun, 
    xmin=-2.31, xmax=2.31,
    ymin=-0.1, ymax=0.1,
    hide y axis,  
    axis equal=true,
  	at={(0,-30)},
  	width=5cm, 
  ]
  \end{axis}
\end{tikzpicture}
}
\end{center}
\end{frame}


%---------------------------------------------------------------------slide----
\begin{frame}
\frametitle{Analysis of functions: relative extrema}
As a consequence of the previous result we can also use the first derivative to determine the relative extrema of a function. 
\begin{theorem}[First derivative test]
Let $f(x)$ be a function with first derivative in an interval $I\subseteq \mathbb{R}$ and let $x_0\in I$ be a stationary point of $f$ ($f'(x_0)=0$).
\begin{itemize}
\item  If $f'(x)>0$ on an open interval extending left from $x_0$ and $f'(x)<0$ on an open interval extending right from $x_0$, then $f$ has a \emph{relative maximum} at $x_0$.
\item  If $f'(x)<0$ on an open interval extending left from $x_0$ and $f'(x)>0$ on an open interval extending right from $x_0$, then $f$ has a \emph{relative minimum} at $x_0$.
\item If $f'(x)$ has the same sign on both an open interval extending left from $x_0$ and an open interval extending right from $x_0$, then $f$ has an \emph{inflection point} at $x_0$.
\end{itemize}
\end{theorem}

\structure{\textbf{Remark}} \emph{A vanishing derivative is a necessary but not sufficient condition for the function to have a relative extrema at a point.}

\structure{\textbf{Example}} The function $f(x)=x^3$ has derivative $f'(x)=3x^2$; it has a critical point at $x=0$.
However it does not have a relative extrema at that point, but an inflection point.
\end{frame}


%---------------------------------------------------------------------slide----
\begin{frame}
\frametitle{Analysis of functions: relative extrema}
\framesubtitle{Example}
Consider again the function $f(x)=x^4-2x^2+1$ and let's analyze its relative extrema now. 
Its first derivative is $f'(x)=4x^3-4x$.
\begin{center}
\tikzsetnextfilename{derivatives_1_variable/extrema_analysis}
\scalebox{0.9}{% Author: Alfredo Sánchez Alberca (asalber@ceu.es)
\begin{tikzpicture}[trim axis left, trim axis right]
  \begin{axis}[
    2dfun, 
    xmin=-2, xmax=2,
    ymin=-2, ymax=2,
    axis equal=true,
    clip=false, 
    width=5cm,
    ]
    \addplot+[domain=-1.5:1.5, smooth, name path=F] {x^4-2*x^2+1} node[anchor=south west] {$f(x)=x^4-2x^2+1$};
    \addplot+[domain=-1.2:1.2, smooth, name path=D] {4*x^3-4*x)} node[anchor=south west] {$f'(x)=4x^3-4x$};
    \node[anchor=east] at (-2,-2.7) {Increase $f(x)$};
    \node[anchor=east] at (-2,-3.5) {Sign $f'(x)$};
    \node[anchor=east, visible on=<2->] at (-2,-4) {Extrema $f(x)$};
    \draw[dashed, gray] (-1,2) -- (-1,-3.5);
    \draw[dashed, gray] (0,2) -- (0,-3.5);
    \draw[dashed, gray] (1,2) -- (1,-3.5);
    \node[color2] at (-1,-3.5) {$0$};
    \node[color2] at (0,-3.5) {$0$};
    \node[color2] at (1,-3.5) {$0$};
    \node[color2] at (-1.5,-3.5) {$-$};
    \draw[->, color1] (-1.5, -2.5) -- (-1.5,-2.8);
    \node[color2] at (-0.5,-3.5) {$+$};
    \draw[->, color1] (-0.5, -2.8) -- (-0.5,-2.5);
    \node[color2] at (0.5,-3.5) {$-$};
    \draw[->, color1] (0.5, -2.5) -- (0.5,-2.8);   
    \node[color2] at (1.5,-3.5) {$+$};
    \draw[->, color1] (1.5, -2.8) -- (1.5,-2.5);
    \fill<3-> (-1,0) circle (1.2pt);
    \node<3->[color1] at (-1,-4) {Min};
    \fill<4-> (0,1) circle (1.2pt);
    \node<4->[color1] at (0,-4) {Max};
    \fill<5-> (1,0) circle (1.2pt);
    \node<5->[color1] at (1,-4) {Min};
  \end{axis};
  \begin{axis}[
    2dfun, 
    xmin=-2.31, xmax=2.31,
    ymin=-0.1, ymax=0.1,
    hide y axis,  
    axis equal=true,
  	at={(0,-30)},
  	width=5cm, 
  ]
  \end{axis}
\end{tikzpicture}
}
\end{center}
\end{frame}


%---------------------------------------------------------------------slide---
\begin{frame}
\frametitle{Analysis of functions: concavity}
The concavity of a function can be determined by de second derivative. 
\begin{theorem}
Let $f(x)$ be a function with second derivative in an interval $I\subseteq \mathbb{R}$.
\begin{itemize}
\item If $\forall x\in I\ f''(x)\geq 0$ then $f$ is concave up (convex) on $I$.
\item If $\forall x\in I\ f''(x)\leq 0$ then $f$ is concave down (concave) on $I$.
\end{itemize}
\end{theorem}

\structure{\textbf{Example}} The function $f(x)=x^2$ has second derivative $f''(x)=2>0$ $\forall x\in \mathbb{R}$, so it is concave up in all $\mathbb{R}$. 
\vskip .5cm
\structure{\textbf{Remark}} \emph{A function can be concave up or down and not have second derivative.}
\end{frame}


%---------------------------------------------------------------------slide----
\begin{frame}
\frametitle{Analysis of functions: concavity}
\framesubtitle{Example}
Let us analyze the concavity of the same function of previous examples $f(x)=x^4-2x^2+1$. 
Its second derivative is $f''(x)=12x^2-4$.
\begin{center}
\tikzsetnextfilename{derivatives_1_variable/concavity_analysis}
\scalebox{0.9}{% Author: Alfredo Sánchez Alberca (asalber@ceu.es)
\begin{tikzpicture}[trim axis left, trim axis right]
  \begin{axis}[
    2dfun, 
    xmin=-2, xmax=2,
    ymin=-4, ymax=2,
    clip=false, 
    width=5cm,
  	]
    \addplot+[domain=-1.5:1.5, smooth, name path=F] {x^4-2*x^2+1} node[anchor=south west] {$f(x)=x^4-2x^2+1$};
    \addplot+[domain=-1.2:1.2, smooth, name path=D] {4*x^3-4*x)} node[anchor=south west] {$f'(x)=4x^3-4x$};
    \addplot+[domain=-0.7:0.7, smooth, name path=D] {12*x^2-4)} node[anchor=south west, pos=0.6] {$f'(x)=12x^2-4$};
    \node[anchor=east, visible on=<2->] at (-2,-4.6) {Concavity $f(x)$};
    \node[anchor=east, visible on=<2->] at (-2,-5.5) {Sign $f''(x)$};
    \draw<3->[dashed, gray] (-0.5773503,2) -- (-0.5773503,-5.3);
    \draw<3->[dashed, gray] (0.5773503,2) -- (0.5773503,-5.3);
    \node<3->[color3] at (-0.5773503,-5.5) {$0$};
    \node<3->[color3] at (0.5773503,-5.5) {$0$};
    \node<4->[color2] at (-1.2,-5.5) {$+$};
    \node<4->[color1] at (-1.2, -4.6) {$\cup$};
    \node<5->[color2] at (0,-5.5) {$-$};
    \node<5->[color1] at (0, -4.6) {$\cap$};
    \node<6->[color2] at (1.2,-5.5) {$+$};
    \node<6->[color1] at (1.2, -4.6) {$\cup$};
    \node[color1, visible on=<7->] at (-0.5773503,-6) {Inflection};
    \node[color1, visible on=<7->] at (0.5773503,-6) {Inflection};
  \end{axis};
  \begin{axis}[visible on=<2->,
  	2dfun, 
 	 xmin=-2, xmax=2,
  	ymin=-0.1, ymax=0.1,
  	hide y axis,  
  	at={(0,-135)},
  	width=5cm, 
  	]
  \end{axis}
\end{tikzpicture}
}
\end{center}
\end{frame}



